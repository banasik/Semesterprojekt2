\chapter{Resumé}

<<<<<<< Updated upstream
Dette projekt beskæftiger sig med hjertesygdommen atrieflimren samt elektrokardiografi. Formålet for projektet er at lave et program, der ud fra en virtuel patients EKG-signal kan afbilde en EKG-graf. Grafen skal være analyserbar i forhold til at detektere atrieflimren hos patienten. Programmet skal desuden kunne gemme EKG-signalets data, både i en privat- og offentlig database. Alt dette er opnået i det kodet program, som er projektets produkt. Analysen er den essentielle del af programmet, som er skrevet på baggrund af viden omkring, hvordan atrieflimren udformer sig som EKG-graf.\\
Vandfaldsmodellen er benyttet til at udvikle softwaren. Metoderne SysML samt UML er benyttet til henholdsvis systembeskrivelse og softwarebeskrivelse. Til beskrivelse af hjertesygdommens patofysiologi, er anvendt metoden tekstanalyse og kildekritik.\\
Programmet lever op til de væsentlige opstillede krav, både funktionelle og ikke-funktionelle. Programmet kan afbilde, analysere og gemme EKG-grafen. Ligeledes er det lykkedes at kunne gemme KEG-signalets data i en privat-, samt offentlig database.\\  
Projektet har altså opfyldt de overordnede krav. Dette er dokumenteret ved en accepttest. Udviklingen af projektet har været præget af projektgruppens vision til virkeligheden, og hvordan projektet i sig selv kunne have indflydelse hos sundhedsprofessionelle.\\
=======
Dette projekt beskæftiger sig med hjertesygdommen atrieflimren samt elektrokardiografi. Formålet for projektet er at lave et program, der ud fra en virtuel patients EKG-signal kan afbilde en EKG-graf. Grafen skal være  analyserbar i forhold til at detektere atrieflimren hos patienten. Programmet skal desuden kunne gemme EKG-signalets data, både i en privat- og offentlig database. Alt dette er opnået i det kodet program, som er projektets produkt. Analysen er den essentielle del af programmet, som er skrevet på baggrund af viden omkring, hvordan atrieflimren udformer sig som EKG-graf. //
Vandfaldsmodellen er benyttet til at udvikle softwaren. Metoderne SysML samt UML er benyttet til henholdsvis systembeskrivelse og softwarebeskrivelse. Til beskrivelse af hjertesygdommens patofysiologi, er anvendt metoden tekstanalyse og kildekritik.\\
Programmet lever op til de væsentlige opstillede krav, både funktionelle og ikke-funktionelle. Programmet kan afbilde, analysere og gemme EKG-grafens data.\\  

En kort konklusion 
>>>>>>> Stashed changes
