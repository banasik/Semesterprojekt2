\chapter{Resumé}

Dette projekt beskæftiger sig med hjertesygdommen atrieflimren samt elektrokardiografi. Formålet for projektet er at lave et program, der ud fra en virtuel patients EKG-signal kan afbilde en EKG-graf. Grafen skal være analyserbar i forhold til at detektere atrieflimren hos patienten. Programmet skal desuden kunne gemme EKG-signalets data både i en privat- og offentlig database. Alt dette er opnået i softwaren, som er projektets produkt. Analysen er den essentielle del af programmet, der er skrevet på baggrund af viden omkring, hvordan atrieflimren udformer sig som EKG-graf.\\
Vandfaldsmodellen er benyttet til at udvikle softwaren. Metoderne SysML samt UML er benyttet til henholdsvis systembeskrivelse og softwarebeskrivelse. Til beskrivelse af hjertesygdommens patofysiologi er anvendt metoden tekstanalyse og kildekritik.\\
Programmet lever op til de væsentlige opstillede krav, både funktionelle og ikke-funktionelle. Programmet kan afbilde, analysere og gemme EKG-grafen. Ligeledes er det lykkedes at kunne gemme EKG-signalets data i en privat- samt offentlig database.\\  
Projektet har dermed opfyldt de overordnede krav, hvilket er dokumenteret ved en accepttest. Udviklingen af projektet har været præget af projektgruppens vision til virkeligheden, og hvordan projektet i sig selv kunne have indflydelse hos sundhedsprofessionelle.\\
\\ \textbf{Abstract}\\
This study examines the heart decease atrial fibrillation together with electrocardiographic signals. The purpose of this project is to code a software to where a virtual patient’s ECG-signal can be plotted on a graph. Said graph has to be analyzable to detect atrial fibrillation in a patient’s heart. The coded software should additionally be adequate at storing the ECG-signal's data in both a private- as well as a public database. All this has been achieved in the program, which is the product of this project. The analysis mechanism of the program is the essential part of the software. This is done by researching atrial fibrillation at its essential features when presented as an ECG-plotted graph.\\
Several methods has been used, including the waterfall model which has been used to set a course for developing the software. UML and SysML has been adopted respectively to the System description and software description. Text analysis and source criticism has benefited to interpret the heart deceases pathophysiology.\\ 
The program complies with both the functional and non-functional requirements. It is capable of illustrating the ECG-signal as well as analyzing and preserving the illustrated graph, in private and public databases.\\
The paper has succeeded its main requirements, which is documented further by an acceptance test. Finally, the progress of this study has been characterized by the concerning group’s vision to the real world, and how this projects topic could have an influence with professionally educated healthcare personal.
