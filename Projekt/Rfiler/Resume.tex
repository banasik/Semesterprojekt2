\chapter{Resumé}

Projektets formål
Dette projekt beskæftiger sig med hjertesygdommen atrieflimren samt elektrokardiografi. Formålet for projektet er at detektere atrieflimren ud fra et EKG, samt at et program skal kunne analysere den givne EKG-graf. Dette er opnået ved et kodet program der er tilknyttet bestemte retningslinier for de EKG-data, som programmet har skulle kunne analysere på. Programmet har desuden til opgave at danne en graf med relevante enheder og data, som en sundhedsprofessionel kan benytte. Vandfaldsmodellen er benyttet til at udvikle softwarekoden. Metoderne SysML samt UML er benyttet til systembeskrivelse softwarebeskrivelse. Til udarbejdelse af hjertesygdommen er metoden tekstanalyse og kildekritik benyttet til disse materialer. Resultatet af projektet har været at vores program har kunnet, ud fra den skrevne kode, afbilde et EKG-signal samt analysere om patienten har atrieflimren. Programmet er desuden også i stand til at gemme de valgte EKG-data i en privat- og offentlig database.
En kort konklusion 
