\subsubsection{Versionhistorik}

\begin{longtabu} to \linewidth{@{}l l l X[j]@{}}
    Version &    Dato &    Ansvarlig &    Beskrivelse\\[-1ex]
    \midrule
    1.0 &   13-05-2015	&    MFJ, LSB og AJF &    Påbegyndt indledning\\
    1.1	&	19-05-2015	&	LSB	&	Færdigskrevet indledning og projektformulering\\
    1.2	&	22-05-2015	&	MCM	& Beskrivelse af projektgennemførelse tilføjet\\
    1.3	&	22-05-2015	&	SSK	&	Tilføjet Specifikation og analyse samt arkitektur\\
    1.4 &	22-05-2015	&	MFJ	&	Tilføjet Fremtidigt arbejde\\
    1.5	&	22-05-2015	&	 LSB &	Tilføjet Systembeskrivelse\\
    1.6	&	25-05-2015	&	SSK	&	Tilføjet Implementering, design og test\\
    1.7	&	25-05-2015	&	MFJ	&	Tilføjet Metode, ordliste og litteraturliste\\
    1.8	&	25-05-2015	&	LSB	&	Tilføjet Resultat\\
    1.9	&	25-05-2015	&	MCM	&	Tilføjet Disskusion\\
    1.10	&	26-05-2015	&	AJF	&	Tilføjet resumé og abstract\\
    1.11	&	26-05-2015	&	CAA	&	Tilføjet krav\\
    1.12	&	26-05-2015	&	ALLE	&	Tilføjet udkast til konklusion\\
    2.0		&	27-05-2015	&	ALLE	&	Gennemrettelse af hele rapporten\\
    	
\label{version_Systemark}
\end{longtabu}

\chapter{Indledning}
I dagens Danmark er incidensen af hjertesygdomme på ca. 45.000 nye tilfælde årligt\footnote{http://www.si-folkesundhed.dk/upload/hjertekarsygdomme\_i\_2011-2\_rapport.pdf}. Mange typer af hjertesygdomme diagnosticeres via et EKG-apparat, der måler patientens hjerteimpulser. Disse impulser bliver afbildet som en graf, der indeholder P-, Q-, R-, S- og T-takker. Det er forholdet mellem disse takker, der fortæller, hvordan hjerteimpulserne hos patienten er. Hvis en patient har et rask EKG-signal, skal forholdet mellem takkerne være indenfor nogle bestemte intervaller. Et EKG-signal, der afviger fra disse standarter siges at være abnormalt og patienten bør tjekkes for en eventuel hjertesygdom. Derfor er et EKG-apparat en vigtig teknologi indenfor sundhedsvæsenet i forhold til videre diagnosticering af hjertesygdomme.\\ \\
Formålet med dette projekt er at udvikle en software, som netop har til formål at detektere en selvvalgt hjertesygdom. Den nødvendige hardware samt et stykke kode, der betragtes som Blackbox, er blevet udleveret ved projektets start.\\
Dette specifikke projekt omhandler sygdommen atrieflimren. En sygdom, der særligt rammer den ældre befolkning, da prævalensen i Danmark er 5-10\%\footnote{https://www.sundhed.dk/borger/sygdomme-a-aa/hjerte-og-blodkar/sygdomme/hjertearytmier/atrieflimren-og-flagren/} for borgere over 80 år.\\
Atrieflimren forekommer, når patientens atrie-kontraktionsmønster forstyrres og dermed begynder at flimre. Atrieflimren karakteriseres ved, at der forekommer 220-300 små udsving pr. minut på EKG-signalets baseline. Desuden vil der i frekvensspektret 300-400 Hz opstå forhøjede amplituder. Det er ud fra denne karakteristik af amplituder, der, i Visual Studio, er blevet udarbejdet en analyse, som kan detektere atrieflimren. Denne analyse er en del af et større program, hvis formål også er at kunne visualisere og gemme et givet EKG-signal i en privat- samt offentlig database. \\ \\
Rapporten er udført som en naturvidenskabelig rapport med først og fremmest et baggrundsafsnit. Dette afsnit fortæller overordnet omkring hjertets funktionalitet, atrieflimren og elektrokardiografi. Det er denne teori, som ligger til grund for systemets design og opbygning samt hvilke krav, der er stillet til systemet.
  
