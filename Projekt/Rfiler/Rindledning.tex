\subsubsection{Versionhistorik}

\begin{longtabu} to \linewidth{@{}l l l X[j]@{}}
    Version &    Dato &    Ansvarlig &    Beskrivelse\\[-1ex]
    \midrule
    Tekst &    Tekst &    Tekst &    Tekst.\\
\label{version_Systemark}
\end{longtabu}

\chapter{Indledning}
I dagens Danmark er incidensen af hjertesygdomme på ca. 45.000 nye tilfælde årligt\footnote{http://www.si-folkesundhed.dk/upload/hjertekarsygdomme\_i\_2011-2\_rapport.pdf}. Mange typer af hjertesygdomme diagnosticeres via et EKG-apparat, der måler patientens hjerteimpulser. Disse impulser bliver afbilledet som en graf, som indeholder P-, Q-, S- og T-takker. Det er forholdet mellem disse takker, der fortæller, hvordan hjerteimpulserne hos patienten er. Hvis en patient har et rask EKG-signal, skal forholdet mellem takkerne være indenfor nogle bestemte intervaller. Et EKG-signal, der afviger fra disse standarter siges at være abnormalt og bør tjekkes for en eventuel hjertesygdom. Derfor er et EKG-apparat en vigtig teknologi indenfor sundhedssystemet i forhold til videre diagnosticering af hjertesygdomme.\\ \\
Formålet med dette projekt er, at udvikle en software, som netop har til formål at detektere en selvvalgt hjertesygdom. Hardwaren til formålet samt et stykke kode, som betragtes som Blackbox, er blevet udleveret ved projektets start.\\
Dette specifikke projekt omhandler sygdommen atrieflimren. En sygdom med en  prævalens, der særligt omfatter den ældre befolkning, da 5-10\%\footnote{https://www.sundhed.dk/borger/sygdomme-a-aa/hjerte-og-blodkar/sygdomme/hjertearytmier/atrieflimren-og-flagren/} af danmarks befolkning over 80 år er ramt af sygdommen.\\
Atrieflimren forekommer, når patientens atriers kontraktionsmønster forstyrres og dermed begynder at flimre. Atrieflimren karakteriseres ved, at der forekommer 220-300 små udsving pr. minut på baselinen. Ud fra denne karakteristik er der i Visual Studio blevet udarbejdet en analyse, der kan detektere atrieflimren. Denne analyse er en del af et større program, der kan visualisere og gemme et et givet EKG-signal.   

  
