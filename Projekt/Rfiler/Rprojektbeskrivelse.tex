\chapter{Projektbeskrivelse}

\section{Projektgennemførelse}
Projektet startede med, at der blev lavet en tidsplan, hvor der var mulighed for ændringer undervejs, dog var der nogle faste deadlines, som skulle følges. De forskellige deadlines ledte op til, at man kunne arbejde efter vandfaldsmodellen, da projektet startede med, at der blev lavet kravspecifikation og accepttest, som beskrev de krav, som programmet skulle kunne udfylde. Derefter var næste deadline, at der skulle laves design, ligesom viste forskellige diagrammer over, hvordan programmet skulle opbygges og hvad der skulle indeholde. Derefter blev programmeret færdiggjort og testet og til sidst finpudset. \\ \\
Altså er der i dette projekt blevet arbejdet efter vandfaldsmodellen, som benyttes, når man arbejder med software, ligesom der er blevet gjort i det pågældende projekt. Vandfaldsmodellen er opbygget på sådan en måde, at man arbejder med de forskellige dele som et vandfald, hvor man tager en af del af gangen og bevæger sig ned gennem de forskellige. De forskellige deadlines vi har haft stemmeroverens med de forskellige led i vandfaldsmodellen, som ses i figur(NUMMER)

\begin{figure}[H]
	\centering
	\includegraphics[width=1\textwidth]{Figurer/Snip20150522_15}
	\caption{Vandfaldsmodel}
\end{figure}

Projektgruppen har været på 8 medlemmer, som er blevet delt ind i 2 grupper, således at arbejdsbyrden blev delt. Den ene gruppe arbejdede med softwareudviklingen, mens den anden gruppe arbejdede med dokumentation og udarbejdelsen af design. Da gruppen har været opdelt, har der været projektmøde hver uge, hvor gruppen har opdateret hinanden og rettet tidsplanen til, hvis det var nødvendigt.


\section{Metode} 
Dette afsnit har til formål, at beskrive hvilke metoder der er benyttet i udarbejdelsen af dette projekt. Primært er der tale om metoder fra faget ISE, samt Sundhedsvidenskab.\\ 
Desuden bliver der i dette afsnit også beskrevet hvilken arbejdsredskaber der er benyttet til udførelse af projekt, rapport og dokumentering.\\ \\
Til beskrivelse, samt opbygning er EKG-målings systemet, er der fra ISE benyttet metoden SysML. SysML er en metode der vha. digrammer analyserer, specificerer, designer og verificerer et givet system. Altså er metoden benyttet til at beskrive systemets opbygning og kommunikationen. Systemets virkning og de hvordan de enkelte dele interagerer er beskrevet ved SD, som er specefik for hver enkelt use case.\\ 
Yderligere er der en applikationsmodel, hvor et samlet overblik over systemet er beskrevet. Denne model består af en domænemodel, hvor alle aktiviteterne i systemet er beskrevet, samt en tilhørende klasse applikationsmodel. Desuden er SD også benyttet til at lave metoderne i applikationsmodellen.\\ 
Softwaren er beskrevet gennem metoden UML, helt specifikt ved et UML klassediagram. Klassediagrammet viser hvilke klasser og metoder al softwaren (med undtagelse af blackbox) består af, samt hvordan systemet er bygget op efter trelagsmodellen.\\
Baggrundsafsnittet, afsnit 3, er udarbejdet ved tekstanalyse og kildekritik, ud fra sundhedsvidenskabelige bøger og IT teori.\\ \\
Af benyttede arbejdsredskaber, er der først og fremmest brugt en fælles arbejdsplatform, GitHub. GitHub er en online platform, hvor der er mulighed for at foretage ændringer samtidigt, og gemme i en fælles mappe, yderlige er der mulighed for en detaljeret versionshistorik. Alle SysML- og UML-diagrammer er udarbejdet i programmet Visio. Koden er skrevet i sproget c\#, i programmet Visual Studio. Visual Studio spiller også sammen med programmet WaveForms generator, i forbindelse med simulering af EKG-signalet. Selve rapporten, mødereferater, logbog og dokumentationen er udformet i tekstprogrammet LaTex. Yderligere er Facebook brugt til mødeindkaldelse og generel kommunikation.



\section{Specifikation og analyse}
I udarbejdelsen af analysen var der mange komplikationer. Dette skyldes primært, at atrieflimren ikke nødvendigvis påvirker et EKG-signal på samme måde, hver gang.\\ \\
Først var der tiltænkt en analyse som skulle tage udgangspunkt i den originale definition for atrieflimren. Måden dette skulle foregå på, var at tage et gennemsnit af baseline, og derefter detektere hvor mange gange, der skete en svingning over baseline. Her skulle der så tjekkes, om svingningerne overtrådte en tærskel. Denne tærskel skulle vurderes ud fra den patofysiologiske baggrund for sygdommen. Efter visualisering af reelle målinger, blev denne metode dog afskrevet, da baseline ikke bliver repræsenteret ved en regulær linje i reelle signaler. \\ \\ 
Derefter blev der udtænkt en metode med en dynamisk baseline. Denne metode viste sig meget tideligt i udviklingsprocessen, til ikke at være kompatibel. Den største komplikation ved denne metode, er at finde en algoritme, kunne udelukke de kendte takker, som karakteriserer et EKG-signal. Hvis denne algoritme ikke blev fundet, ville den dynamiske baseline altid ligge et stykke over den reelle baseline, grundet de høje R-takker. \\ \\
De to første metoder blev aldrig færdiggjort, da problemerne opstod, efter pseudokode begyndte at blive udarbejdet. Herefter gik gruppen til vejleder for at finde en alternativ løsning til analysen. Vejleder fik herefter input fra en anden professionel, og kunne derefter hjælpe med at udarbejde en analyse, som virkede.\\ \\
Vejleder kunne oplyse at atrieflimren har specifikke kendetegn, hvis der bliver analyseret på EKG-signalets amplituder inden for specifikke frekvenser, og det er ud fra denne information, at den endelige analyse blev udarbejdet. 

\section{Arkitektur}
Softwaren er bygget op i henhold til trelagsmodellen, hvor GUI’erne fungerer som programmets brugerinterface, med et login-vindue, et CPR-vindue og et EKG-vindue. Her fungerer EKG-vinduet, som det primære vindue, hvor EKG-signaler kan måles. Grafiklaget er yderligere beskrevet i dokumentationen, men kan også visualiseres ud fra figur XX nedenfor.

\begin{figure}[H]
	\centering
	\includegraphics[width=1\textwidth]{Figurer/Snip20150512_9}
	\caption{UML-klassediagram}
\end{figure}
 
Logiklaget kan ses som kernen i softwaren. Det er her alt data bliver behandlet fra datalaget, samt videresender data fra målinger til datalaget. Det er blandt andet her analysen af målingen sker, og hvor indtastede oplysninger bliver valideret. Logiklaget fungerer som bindeleddet mellem de data, som kommer fra datalaget, og GUI’erne. \\ \\
Datalaget er til, for primært at håndtere forbindelsen med hardwaren og databaserne. I denne klasse bliver der skabt forbindelse til både SQL og DAQ. Datalaget henter data fra den private database, som logiklaget bruger til validering. Denne klasse gemmer også data givet fra logiklaget i både den private- og offentlige database. Klassen logiklag og datalag er beskrevet yderligere i dokumentationen. \\ \\
For at beskrive koden yderligere, er der lavet en domænemodel. Domænemodellen repræsenterer hele koden, altså hvordan flowet imellem klasserne går, og hvilken overordnet kommunikation, der foregår. Alle vinduerne repræsenterer GUI’er og alle tabeller er tabellerne i den private database. Domænemodellen kan ses på figur XX. 

\begin{figure}[H]
	\centering
	\includegraphics[width=1\textwidth]{Figurer/Snip20150525_18}
	\caption{Domænemodel}
\end{figure}


\subsection{Design}
Alle tanker omkring, hvordan koden skulle være bygget op, bunder i trelagsmodellen. Der er derfor sørget for at der ikke er noget kode, som kommunikerer med et lag, som de ikke har tilladelse til. Der kan læses mere om trelagsmodellen i dokumentationen(henvisning).\\ \\
I dette projekt, har der i slutfasen været fokuseret meget på brugervenlighed og feedback til brugeren. Tideligt i projektet, var det ikke tydeligt, hvornår der blev foretaget en måling. Dette er blevet tydeliggjort, ved at musen bliver til en cirkel, der er i bevægelse, imens der tages en måling. Der er også lavet et pop-up vindue, som bekræfter, at en måling er blevet gemt.\\ \\
Der er desuden, lavet en feature, som viser grid-lines på grafen. I gruppen, blev der besluttet, at der skulle være to typer af grids. En lille og en stor type.  De store grids, repræsenterer hver 0,2 sekunder og de små, svarer til 0,04 sekunder. Disse intervaller er fastlagt, ud fra standarder for professionelle EKG-displays.  Dette blev herefter implementeret i koden.\\ \\
En anden ting, som har været meget dominerende i overvejelserne, er sikkerhed. Det er personfølsomme oplysninger, som skal detekteres i dette program, er det er derfor vigtigt at det ikke er alle og enhver, der skal kunne få adgang til det. Derfor er der et vindue, som beder brugeren om at logge ind, med et gyldigt brugernavn og kodeord. Dette vil validere, at denne person har rettigheder, til at få adgang til systemet. En valideret person, vil typisk være en sundhedsprofessionel.\\ \\
Yderligere er der lavet et identifikationsvindue, hvor der skal indtastes et CPR-nummer. Ved hjælp af denne oplysning, bliver der hentet en person. Denne person bliver brugt igennem systemet, som indikator for, hvem den her person er. Navnet bliver f.eks. vist i EKG-vinduet, og det er også dette nummer der bliver brugt, når målingerne skal gemmes.  

\subsection{Implementering}
Analysen er det essentielle i implementeringen. Hvis ikke der er en analyse der fungerer, så programmet obsolet. I et tideligere afsnit er der beskrevet, at analysen blev bygget op omkring en anden definition af atrieflimren, end den patofysiologien beskriver.
\\
Analysen er i stedet bygget op omkring tre for-løkker, som identificerer forhøjede amplituder inden for et specielt frekvensspektrum. Ved hjælp af matematikbiblioteket alglibnet2, bliver signalet først konverteret til et komplekst Furier-transformeret array, som indeholder koordinater repræsenterende vektorer for amplituden i signalet. 

\begin{figure}[H]
	\centering
	\includegraphics[width=0.8\textwidth]{Figurer/Snip20150525_41}
	\caption{Kode udsnit af analyse implementering}
\end{figure}

Herefter bliver amplituderne regnet ud, ved hjælp af standard formelen for vektorudregning. Disse værdier bliver tilføjet til en ny liste. 

\begin{figure}[H]
	\centering
	\includegraphics[width=1\textwidth]{Figurer/Snip20150525_43}
	\caption{Kode udsnit af analyse implementering}
\end{figure}

Herefter bliver arrayet udspecificeret til kun, at indeholde de pladser, som repræsenterer amplituderne for det valgte frekvensspektrum. 

\begin{figure}[H]
	\centering
	\includegraphics[width=0.7\textwidth]{Figurer/Snip20150525_44}
	\caption{Kode udsnit af analyse implementering}
\end{figure}

Disse værdier bliver derefter tjekket for, om de indeholder en amplitude, som ligger over tærskelværdien. Hvis dette er tilfældet, returnere metoden ’true’, hvilket repræsenterer at det er rigtigt, at denne person kan have atrieflimren. Hvis der ikke findes en værdi, som er højere end tærsklen, returnerer metoden ’false’. 

\begin{figure}[H]
	\centering
	\includegraphics[width=0.8\textwidth]{Figurer/Snip20150525_47}
	\caption{Kode udsnit af analyse implementering}
\end{figure}

Tærskelværdien er blev fundet ud fra testprogrammet, som der kan læses yderligere om i dokumentationen(henvisning). Der kan desuden også læses yderligere specifikation omkring implementeringen i dokumentationen(henvisning). 


\subsection{Test}
I dette projekt er der hverken lavet modul eller integrationstest. I stedet er koden blevet testet efterhånden, som metoder er blevet færdigt gjort. Det er også på denne måde, at gruppen har kunnet tjekke, at metoden ikke melder fejl, eller får systemet til at bryde sammen. Metoden "KørEKG" kan give et eksempel på hvordan forløbet har været. "KørEKG" er en essentiel del af koden, og det var derfor vigtigt, at den fungerede fra starten af. Her blev den estimerede kode først skrevet, og der blev derefter kørt et EKG for at se, om grafen kom frem. Dette gjorde den ikke i første omgang, og der blev derfor tilføjet en linje kode, som henter den maksimale volt, som kommer igennem Analog Discovery. 

\section{Resultater og diskussion}
Kravene til dette projekt er at afbillede, analysere og gemme EKG-signaler fra virtuelle patienter. Alt dette er lykkes. \\ 
Før det er muligt at starte en måling skal man igennem et Login-vindue samt et CPR-vindue, hvor man indtaster den virtuelle patients CPR-nummer. Når login og CPR-nummeret til blevet godkendt vises EKG-vinduet, som er vinduet, hvor programmet køres fra. \\ \\
Ved tryk på "Start ny måling" går der X-antal minutter og EKG-signalet bliver afbilledet som en EKG-graf. Se figur NUMMER. 

\begin{figure}[H]
	\centering
	\includegraphics[width=1\textwidth]{Figurer/Snip20150525_25}
	\caption{Visning af EKG-signal samt analyse af Sundt EKG-signal}
\end{figure}

Når EKG-grafen vises i EKG-vinduet har programmet også analyseret EKG-signalet i forhold til atrieflimren. Hvis EKG-signalet er normalt skriver programmet "Sundt EKG" under Resultat, se figur Nummer (ovenfor). Hvis EKG-signalet afviger fra standart værdierne skriver programmet "Tjek for Atrieflimmer!!", se figur Nummer. 

\begin{figure}[H]
	\centering
	\includegraphics[width=1\textwidth]{Figurer/Snip20150525_26}
	\caption{Analyse af sygt EKG-signal}
\end{figure}

Efter visningen samt analysen af EKG-signalet skal data om målingen gemmes i en privat- samt en offentlig database. Dette sker ved tryk på "Gem ny måling". Et pop-up vindue fremkommer og bekræfter handlingen. Efterfølgende kan man i den private database under "Måling"\--tabellen, se de gemte måleringer, se figur nummer.   

\begin{figure}[H]
	\centering
	\includegraphics[width=1\textwidth]{Figurer/Snip20150525_27}	
	\caption{Lagring af data i privat database}
\end{figure}

I den offentlige data base er der to tabeller. Den ene hedder EKG\_Data. Her bliver den virtuelle patientens og brugerens data gemt, se figur nummer. Den anden hedder EKG\_måling. Her bliver informationer omkring selve målingen gemt, se figur nummer. 

\begin{figure}[H]
	\centering
	\includegraphics[width=1\textwidth]{Figurer/Snip20150525_28}
	\caption{Lagring af den virtuelle patients- og brugerens data i den offentlig database}
\end{figure}

\begin{figure}[H]
	\centering
	\includegraphics[width=1\textwidth]{Figurer/Snip20150525_29}
	\caption{Lagring af målingens data i offentlig database}
\end{figure}

Med hensyn til visningen af målingen, kunne der godt have været et mere præcis gitter, således at lægerne nemmere kunne aflæse ud fra grafen om patienten har atrieflimmer. Det kan godt lade sig gøre at aflæse på nuværende tidspunkt, men det er mindre præcis, da ternene ikke er kvadratiske.

Det har ikke været muligt at gøre analysen så præcis, at den kunne diktere atrieflimmer helt præcis, men derimod gør den lægen opmærksom på, at der er en unormalitet i forhold til en normal EKG. Analysen kunne ikke gøres præcis, da der findes mange forskellige typer af atrieflimmer, fordi mennesker er forskellige.

Da det kom sent ud, at man skulle kunne gemme i en offentlig database, er flere af værdierne blev fravalgt pga. tidspres, men man kunne have indført nogle flere værdier, så man f.eks. kunne ligge patientens CPR-nummer og navn ind og gemme det i EKG\_Data.


\section{Opnået erfaringer}
\textbf{Lise Skytte Brodersen}

\textbf{Mads Fryland Jørgensen}

\textbf{Albert Jakob Fredshavn}

\textbf{Malene Cecilie Mikkelsen}

\textbf{Mohammed Hussein Mohamed}

\textbf{Sara-Sofie Staub Kirkeby}
I dette projekt, har jeg primært arbejdet med det programmeringsmæssige aspekt. Personligt, syntes jeg det har været en anelse frustrerende, da der i den sammenhæng, er dukket nogen ting op, som ikke har været en del af vores undervisning, i dette semester. Det har krævet meget tænkning ud af boksen. Dette har selvfølgeligt også gjort, at vi har været tvunget til, at lære os selv, yderligere ting. 
Jeg har siddet med det overordnede ansvar for analysen, og det viste sig at være mere problematisk, end først regnet. Analysen blev ændret tre gange undervejs, og hver eneste gang, har der været behov for, at skrive analysen fuldstændig om. Her har vejleder dog været rigtigt god til, at træde til, og give et andet syn på, hvordan analysen kunne laves, og det hjalp meget. 
Et andet problem har været, at vi valgte at dele gruppen op i to; en der lavede det meste tekstarbejde, og en anden del, som stod for programmeringen. Kommunikationen imellem de to undergrupper, har til tider ikke været særligt godt, og det har betydet at vi er blevet nødt til at ændre nogen ting undervejs. 
Jeg mener generelt at jeg har lagt et godt stykke arbejde i gruppen, og jeg har forsøgt at gøre så meget jeg overhovedet kunne. Jeg har lært meget om EKG-signaler og deres karakteristikker, samt omkring, hvordan sådan et program skal programmeres. Jeg kunne dog eventuelt godt have forsøgt at tage mere initiativ, og have taget lidt mere kontakt med skriveholdet. 
Projektet er efter min mening gået rigtigt godt, og jeg mener at det er et stykke arbejde vi godt kan være stolte af.

\textbf{Martin Banasik}

\textbf{Cecilie Ammitzbøll Aarøe}
Igennem dette projekt arbejdede jeg i den gruppe som bla. udarbejdede projektdokumentationen. Det gjorde at jeg fik mulighed for at arbejde med nogle af de værktøjer, som vi har lært i ISE. En ulempe har dog været, at jeg dermed ikke var med til at programmere programmet. Den del af gruppen, som programmerede, var dog meget gode til at udkommantere programmering, så vi andre nemt kunne sætte os ind i den. Desuden afholdte vi enkelte møde, hvor programmeringen blev gennemået i de mindste detaljer. Men det er stadigvæk ikke det samme som at sidde med det selv.
Jeg synes gruppen har arbejdet godt sammen, og vi havde fra starten afstemt vores forventninger til projektet. At disse forventninger så var lidt højere end det var muligt at efterkomme, er noget der altid vil ske i starten af et projekt. Men jeg vil hellere hav store forventninger til et projekt og så afstemme dem med hvad der er realistisk, frem for at sætte baren lavt fra starten.
Vi havde også lidt kommunikations vanskeligheder i starten af projektet. Dokumentationsgruppen skulle designe systemet udfra funktionelle og ikke-funktionelle krav. Vi rådførte os med programmeringsgruppen, men der opstod en misforståelse mellem grupperne, som gjorde atvi ikke helt havde forstået hvad systemet helt præcist gjorde.
Men alt i alt synes jeg, det har været en god oplevelse at lave dette projekt. Gruppen har fungeret rigtig godt. Der har været en god stemning, og vi har været gode til at hjælpe hinanden, når der var brug for det.

\section{Fremtidigt arbejde}
Som følge af at projektet er tiltænkt som en prototype, er der løbende gennem projektudførelsen opstået en masse muligheder og idéer for videreudvikling af systemet. \\\\
Den første helt basale idé, som også er forsøgt udført sideløbende i projektet, er etablering af en "opret ny patient" funktion. Funktionen skal muliggøre, at den sundhedsprofessionelle kan oprette en ny patient i systemet, i forbindelse med indtastning af patientens CPR-nummer. Funktionen skal fungere således, at hvis ikke det indtastede CPR-nummer i forvejen er kendt i systemet, skal skridtet efter CPR-vinduet være et nyt "Opret Patient"-vindue. Her skal den sundhedsprofessionelle kunne indtaste relevante oplysninger omkring patienten, og til slut oprette patienten i både den private- og offentlige database.\\\\
Et ideelt område til videreudvikling er brugervenlighed, både på software plan og i særhed på hardware plan.\\
Software kun udvikles i en retning, hvor det bliver lettere og mere overskueligt for den sundhedsprofessionelle, at analysere og evaluere EKG-signalet. En forbedring ville være, at der skal være mulighed for at trække en eller flere x- og y-cursors ned over EKG-signalet, og dermed få vist amplitude og tid, og relevante intervaller. Herefter kunne en mulighed være, at de observerede værdier kunne gemmes som tilhørende tekst til det specifikke EKG-signal. \\
Sideløbende, imens EKG-målingen foretages, vil det være muligt, at have direkte adgang til den pågældende patients sygejournal. Adgangen til sygejournalen skal kunne læse i en et andet vindue, som er synligt samtidig med EKG-vinduet arbejder, hvorefter det er muligt at skifte mellem disse vinduer, og tilføje ændringer, notater etc. i journalen. Med andre ord skal systemet understøtte EPJ\\ \\
Den endelige udgave af softwaren skal implementeres på et mere brugervenligt interface, eksempelvis en tablet eller lignende. Samtidig skal systemet være tilknyttet en håndholdt EKG-måler, i form af en holter, hvorefter softwaren aflæser data fra, og udskriver på tabletten. Desuden skal det være muligt for den sundhedsprofessionelle at vælge indstillinger, alt efter hvad der ønskes analyseres for. Det endelige produkt er kun tilpasset analyse for atrieflimren, men det skal være muligt at kunne vælge, undersøgelse for eksempelvis andre sygdomme som ventrikelflimren, ST-elavation osv. De forskellige undersøgelser skal derudover også kunne mikses på kryds og tværs, hvis patienten har blandede symptomer, således at der søges for flere sygdomme. Dette vil medføre, at den sundhedsprofessionelle kan tage tage udstyr med sig på hjemmebesøg, såvel som at patienten selv kan foretage en måling.

