\chapter{Projektbeskrivelse}

\section{Projektgennemførelse}

\section{Metode}

\section{Specifikation og analyse}

\section{Arkitektur}

\subsection{Design}

\subsection{Implementering}

\section{Resultater og diskussion}

\section{Opnået erfaringer}

\section{Fremtidigt arbejde}
Som følge af at projektet er tiltænkt som en prototype, er der løbende gennem projekt udførelsen opstået en masse muligheder og ideer for videreudvikling af systemet. \\
Den første helt basale idé, som også er forsøgt udført sideløbende i projektet, er etablering af en "opret ny patient" funktion. Funktionen skal muliggøre, at den sundhedsprofessionelle kan oprette en ny patient i systemet, i forbindelse med indtastning af patientens CPR nummer. Funktionen skal fungere således, at hvis ikke det indtastede CPR nummer i forvejen er kendt i systemet, skal skridtet efter CPR-vinduet være et nyt "Opret Patient"-vindue. Her skal den sundhedsprofessionelle kunne indtaste relevante oplysninger omkring patienten, og til slut oprette patienten i både den private- og offentlige database.\\
Et ideelt område til videreudvikling er brugervenlighed, både på software plan og i særhed på hardware plan.
