\chapter{Konklusion}
I dette projekt, er der blevet udviklet en software prototype, som kan afbilde, analysere og gemme EKG-signaler. \\\\
Gruppen startede ud med høje ambitioner, omkring prototypen, men opdagede hurtigt, at tankerne omkring, hvordan prototypen ideelt set skulle være, og hvad der reelt kunne udvikles ikke stemte overens. \\
På trods af de høje forventninger har gruppen formået at opfylde alle overordnede krav, som blev sat i starten af projektet. Dette kan ses på den gennemførte accepttest, som kun havde en enkelt designrelateret afvigelse. Kravene om at afbilde og analysere et EKG-signal er blevet opfyldt, og det er også lykkedes gruppen at implementere en funktion, som kan gemme i en privat database. 
Gruppen har oplevet at, det har været en udfordring at udvikle en analyse, som kunne dække over varierende signaler. Det blev klart, at et EKG-signal diagnosticeret med atrieflimren kan variere i udseende, og det blev, derfor svært for gruppen at udvikle en passende algoritme. Der blev derfor udarbejdet en algoritme, ud fra et specifikt signal, dog passer denne algoritme også på et sundt EKG-signal, således at det erklæres sundt. 
Kravet fra Sundhedsstyrelsen, om en offentlig database, blev præsenteret sent i arbejdsprocessen, men det lykkedes gruppen at implementere det, uden store vanskeligheder. 
Udviklingsprocessen har været præget, af at gruppen har tænkt meget over, hvordan dette system ville kunne fungere ude i virkeligheden. Dette kan ses på nogle af de krav, som gruppen selv har sat til projektet. 
Projekt er overordnet set, gået godt, og gruppen er tilfreds, med slutproduktet.
undtagen Lise, hun er en flise. 
