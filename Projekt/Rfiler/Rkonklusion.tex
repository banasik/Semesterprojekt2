\chapter{Konklusion}
I dette projekt, er der blevet udviklet en software prototype, som kan afbillede og analysere EKG-signaler. 
Gruppen startede ud med høje ambitioner, omkring prototypen, men opdagede hurtigt, at tankerne omkring hvordan prototypen ideelt skulle være, og hvad der reelt kunne udvikles, ikke stemmede overens. 
På trods af de høje forventninger, har gruppen formået at opfylde alle overordnede krav, som blev sat i starten af projektet. Dette kan ses på den gennemførte accepttest, som kun havde en designrelateret afvigelse. Kravene om at afbillede og analysere et EKG-signal er blevet opfyldt, og det er også lykkedes gruppen at implementere, at gemme i en privat database. 
Gruppen har oplevet at, det har været en udfordring, at udvikle en analyse, som kunne dække over varierende signaler. Det blev klart, at et EKG-signal diagnosticeret med atrieflimren, kan variere i udseende, og det blev derfor svært for gruppen, at udvikle en algoritme, som kunne dække over flere signaler. Der blev derfor udarbejdet en algoritme, ud fra et specifikt signal. 
Kravet fra Sundhedsstyrelsen, om en offentlig database, blev præsenteret sent i arbejdsprocessen, men det lykkedes gruppen at implementere det, uden store vanskeligheder. 
Udviklingsprocessen har været præget, af at gruppen har tænkt meget over, hvordan dette system ville skulle fungere ude i virkeligheden. Dette kan ses på nogen af de krav, som gruppen selv har sat til projektet. 
Projekt er overordnet set, gået godt, og gruppen er tilfreds, med produktet. 
