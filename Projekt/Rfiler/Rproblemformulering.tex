\chapter{Projektformulering}

Dette projekt har til formål at udvikle software, der kan afbilde, detektere atrieflimmer samt gemme et EKG-signal i en database. Projekt tager udgangspunkt i virtuelle patienters EKG-signaler. Disse EKG-signaler hentes fra Physionet\footnote{www.physionet.org}, som er en ekstern database, der indeholder mange forskellige EKG-signaler fra forskellige patienter. EKG-signalerne  ændres fra en CSV-fil til et digitalt signal af udleveret hardware, som består af Analog-discovery og en DAQ. \\
Det er dette digitale signal softwaren skal kunne afbilde som en graf - en såkaldt EKG-graf. Et andet krav til projektet er, at softwaren skal via en analyse detektere atrieflimren hos den virtuelle patient. Målingen skal også gemmes i en privat database. Den private database skal forståes således, at databasen er tilknyttet et specifikt sted, fx et sygehus, hvor lige præcis denne EKG-måling er foretaget. \\
Den 4. Maj 2015 indførte Sundhedsstyrelsen et krav, om at alle EKG-målinger foretaget i Danmark skal gemmes i en offentlig database. Det vil sige, at EKG-signalet skal både gemmes i den private- og den offentlige database. Den offentlige database gør det muligt at kunne tilgå et EKG-signal uafhængig af, hvor målingen er foretaget.\\
Softwaren skal udarbejdes i Visual Studio på baggrund af trelagsmodellen\footnote{Se dokumentation 3.3}.\\ \\
Så helt specifikt skal dette projekts produkt være en prototype til et software system, som kan benyttes i et EKG-apparat. Softwaren kan afbilde EKG-grafen, gemme EKG-signalets data i privat- og offentlig database, samt detektere forekomsten af atrieflimren for den virtuelle patients EKG-signal.  


\textbf{Ansvarsområde}
\begin{longtabu} to \linewidth{@{}l  l X[j]@{}}
    Dokument &    Afsnit &    Ansvarlig\\[-1ex]
    \midrule
    Rapport &   Indlednig	&    MFJ, LSB, CAA og AJF\\

\end{longtabu}