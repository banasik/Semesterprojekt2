\chapter{Projektformulering}

Dette projekt har til formål at udvikle software, der kan afbillede, detektere atrieflimre samt gemme et EKG-signal i database. Projekt tager udgangspunkt i virtuelle patienters EKG-signaler. Disse EKG-signaler hentes fra Physionet\footnote{www.physionet.org}, som er en ekstern database, der indeholder mange forskellige EKG-signaler fra forskellige patienter. EKG-signalerne behandles af udleveret hardware, som består af Analog-discovery og en DAQ. Det hentet signal ændres fra en CSV-fil til et digitalt signal. \\
Det er dette digitale signal softwaren skal kunne afbillede som en graf - en såkaldt EKG-graf. Et andet krav til projektet er, at softwaren skal via en analyse detektere atrieflimren hos den virtuelle patient. Målingen skal også gemmes i en privat database. Den private database skal forståes således, at databasen er tilkoblet et specifikt sted, fx et sygehus, hvor lige præcis denne EKG-måling er foretaget. \\
Den 4. Maj 2015 indførte Sundhedsstyrelsen et krav, om at alle EKG-målinger foretaget i Danmark skal gemmes i en offentlig database. Det vil sige, at EKG-signalet skal både gemmes i den private- og den offentlige database. Den offentlige database gør det muligt at kunne tilgå et EKG-signal uafhængig af, hvor målingen er foretaget.\\ \\
Så helt specifikt skal dette projekts produkt være et software system, der kan afbillede EKG-graf, gemme EKG-signal i privat- og offentlig database samt detektere forekomsten af atrieflimmer for den virtuelle patients EKG-signal.  