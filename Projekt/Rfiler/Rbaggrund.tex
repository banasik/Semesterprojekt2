\chapter{Baggrund}

\section{Hjertet}
Hjertet, \textit{cor}, er en hul muskel, der har til opgave at pumpe blodet rundt til hele kroppen. Hjertet består af i alt fire kamre. To forkamre, atrier og to hjertekamre, ventrikler. Atrierne fungere primært som reservoir for blod, mens ventriklerne fungerer som den effektive pumpe.\\
Hjertekamrene og forkamrene er adskilt fra hinanden af anulus fibrosus, som er en plade af bindevæv. Anulus fibrosus består af fire bindevævsringe, der er forbundet med hinanden. To af disse udgør åbningerne mellem atrierne og ventriklerne. De to sidste danner åbningerne mellem højre hjertekammer og lungepulsåren og venstre ventrikel og hovedpulsåren. Ved alle bindevævsringene er der klapper, der fungere som ventiler.\\ 
AV-klapperne \textbf{ORDLISTE, atrioventrikulær - klapperne} sidder mellem atrierne og ventriklerne. Klappen mellem højre atrier og ventrikel kaldes tricuspidalklap, mens klappen mellem venstre atrier og ventrikel kaldes mitralklap. Aortaklappen er placeret ved afgangen af hovedpulsåren og pulmonalklappen ved afgangen af lungepulsåren. Klapperne fungere således, at blodet kun kan løbe én vej gennem dem. Åbningen samt lukningen af disse er en passiv proces, som bestemmes af forskelle i væsketrykket på de to sider af klapperne.\\ 
\begin{figure}[htb]
	\centering
	\includegraphics[width=1\textwidth]{Figurer/Snip20150410_31}
	\caption{Hjerte med forklarende pile \protect\footnotemark} 
\end{figure}
\footnotetext{Billede fra Hjerteforeningens hjemmeside.\textbf{Indsæt hyperlink - ligger i en note på Lises computer}}



(Når trykket i ventriklerne bliver mindre end trykket i atrierne, åbnes AV-klapperne. På den måde strømmer blodet passivt ind i ventriklerne. Atrierne kontraherer sig og presser mere blod ned i ventriklerne, så atrierne tømmes for blod. Ventriklerne begynder at trække sig sammen, således at ventrikeltrykket \textbf{ Muligvis et billede} )\\ \\
Hjertets cyklus inddeles i to hovedfaser. Den første kaldes diastolen. I diastolen er ventriklerne afslappede og fyldes med blodet. Det vil sige, at trykket i ventriklen bliver lavere end trykket i atrierne, således at AV-klappen åbnes, og blodet begynder at strømme ind i ventriklen. Under hele diastolen er aortaklappen lukket. Den anden fase kaldes systolen. I systolen kontraherer ventriklerne sig. Trykket i ventriklen overstiger trykket i atrierne, således at AV-klapperne lukkes, så tilbagestrømning af blod til atrierne forhindres. Når ventriklen har kontraheret sig så meget, at trykket i ventriklen overstiger trykket i hovedpulsåren, åbnes aortaklappen, og blodet strømmer ud i aorta. Ventriklernes tryk falder igen til under atriernes tryk, hvilket påvirker at AV-klapperne åbnes igen og diastolen begynder igen.\\
Hjertets cyklus igangsættes i sinusknuden ved aktionspotentialer, der føres til de forskellige dele af hjertet. Dette sker enten ved at aktionspotentialet går fra hjertemuskelcelle til hjertemuskelcelle gennem åbne celleforbindelser. Eller genne åbne celleforbindelser mellem celler i hjertets specielle ledningssystem, der består af specialiserede hjertemuskelceller. Det specielle ledningssystem består af tre sammenhængende dele - AV-knuden \textbf{ Ordliste}, det hiske bundt gennem anulus fibrosus og det hiske bundt over i purkinjefibrene \textbf{ ordliste}. \\ \\

\begin{figure}[htb]

	\centering	
	\includegraphics[width=1\textwidth]{Figurer/Snip20150410_6}
	\caption{Spredning af aktionspotentialer gennem hjertet \protect\footnotemark}
\end{figure}
\footnotetext{"Menneskets anatomi og fysiologi" s. 275 figur 9.9}







  