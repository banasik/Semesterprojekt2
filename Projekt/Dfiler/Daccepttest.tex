\chapter{Acceptest}

\section{Accepttest af Use Cases}


%Use case 1 acceptest
\subsection{Use Case 1}
\textbf{Log ind}

\begin{longtabu} to \linewidth{@{} c X[j] X[j] X[j] l@{}}
    ~ &	Test &    Forventet resultat &		Faktiske observationer &    Godkendt\\[-1ex]
    \midrule
    ~ &\textit{Hovedscenarie} & ~ & ~ &
    \\ \midrule
    1. &Indtast username samt password &   Username- og passwordboks bliver udfyldt  &    ~ &		%{\Huge \checkmark}
    \\
    2. &Tryk på "Login"-knappen.  &    Login bliver godkendt. Login-vinduet lukkes ned mens EKG-vinduet åbnes  &     ~ &		%{\Huge \checkmark}
	\\ \midrule
	~ &\textit{Exentions} & ~ & ~ & 
	\\ \midrule	
    2a. &	Username eller password er forkert &    Messageboks vises på skærmen med teksten "Username eller password er forkert - prøv igen"  &   ~  &		%{\Huge \checkmark}
 \\ \bottomrule
 
\caption{Accepttest af Use Case 1.}\\
\label{AT_UC1}
\end{longtabu}

%Use case 2 acceptest 
\subsection{Use Case 2}
\textbf{Vis EKG}

\begin{longtabu} to \linewidth{@{} c X[j] X[j] X[j] l@{}}
	& Test	& Forventet resultat		& Faktiske observationer		& Godkendt\\[-1ex] 
	\midrule
	&\textit{Hovedscenarie} & & & 
	\\ \midrule
	1. &Vælg indstillinger & Indstillinger bliver valgt & & %{\Huge \checkmark}
	\\
	2. & Tryk på "Start" & Målingen startes i EKG-vinduet & & %{\Huge \checkmark}
	\\
	3. & EKG-data illustreres på en graf & En analyserebar graf fremvises i EKG-vinduet & & %{\Huge \checkmark}
	\\ \midrule
	1.a & Ingen ændring i indstillinger  & Målingen foretages med default-indstillingerne & & %{\Huge \checkmark}
	\\ \bottomrule

\caption{Accepttest af Use Case 2.}\\
\label{AT_UC2}	
\end{longtabu}

%Use Case 3 acceptest

\subsection{Use Case 3}
\textbf{Evaluer EKG}

\begin{longtabu} to \linewidth{@{} c X[j] X[j] X[j] l@{}}
    ~ &	Test &    Forventet resultat &		Faktiske observationer &    Godkendt\\[-1ex]
    \midrule
    ~ &\textit{Hovedscenarie} & ~ & ~ &
    \\ \midrule
    1. & Validere programmets analyse af EKG-signalet &    Det er muligt at se små fluktuationer, som kan aflæses på EKG-grafen  &    ~ &		%{\Huge \checkmark}
    \\
    2. &Stil diagnosen atrieflimmer	 &    Atrieflimmer kan aflæses ud fra EKG-grafen  &     ~ &		%{\Huge \checkmark}
	\\ \midrule
	~ &\textit{Exentions} & ~ & ~ & 
	\\ \midrule	
    2a. &	Atriefrekvensen er ikke i intervallet 220-300 pr. minut &    Det er ikke muligt at diagnosticere atrieflimmer ud fra EKG-grafen   &   ~  &		%{\Huge \checkmark}
 \\ \bottomrule
 
\caption{Accepttest af Use Case 3.}\\
\label{AT_UC3}
\end{longtabu}

%Use Case 4 acceptest

\subsection{Use Case 4}
\textbf{Gem EKG}

\begin{longtabu} to \linewidth{@{} c X[j] X[j] X[j] l@{}}
    ~ &	Test &    Forventet resultat &		Faktiske observationer &    Godkendt\\[-1ex]
    \midrule
    ~ &\textit{Hovedscenarie} & ~ & ~ &
    \\ \midrule
    1. &Tryk på "Gem"-knappen. &    Information-vinduet kommer frem, og der er mulighed for at indtaste information om målingen  &    ~ &		%{\Huge \checkmark}
    \\
    2. &Indtaster dens fiktive persons CPR-nummer, dato og diagnose for den givende måling	 &   CPR-nummer-, dato- og diagnoseboks bliver udfyldt &     ~ &		%{\Huge \checkmark}
    \\
    3.	&Tryk på "Gem"-knappen i informaiton-vinduet 	&	Messageboks kommer frem med teksten "Målingen er gemt" & & %{\Huge \checkmark}
	\\ \midrule
	~ &\textit{Exentions} & ~ & ~ & 
	\\ \midrule	
    2a. &	CPR-nummer er ikke gyldigt &    Messageboks vises med teksten "CPR-nummer er ikke gyldig". CPR-nummerboks cleares   &   ~  &		%{\Huge \checkmark}
 \\ \bottomrule
 
\caption{Accepttest af Use Case 4.}\\
\label{AT_UC4}
\end{longtabu}

%Use Case 5 acceptest

\subsection{Use Case 5}
\textbf{Log ud}

\begin{longtabu} to \linewidth{@{} c X[j] X[j] X[j] l@{}}
    ~ &	Test &    Forventet resultat &		Faktiske observationer &    Godkendt\\[-1ex]
    \midrule
    ~ &\textit{Hovedscenarie} & ~ & ~ &
    \\ \midrule
    1. & Tryk på "log ud"-knappen &    EKG-vinduet lukkes ned, mens login-vinduet fremkommer &    ~ &		%{\Huge \checkmark}
   	\\ \midrule
	~ &\textit{Exentions} & ~ & ~ & 
	\\ \midrule	
 \\ \bottomrule
 
\caption{Accepttest af Use Case 5.}\\
\label{AT_UC5}
\end{longtabu}

\section{Accepttest af ikke-funktionelle krav}

\begin{longtabu} to \linewidth{@{} c X[j] X[j] X[j] X[j] l@{}}
	& Ikke-funktionelt krav & Test/handling & Forventet resultat & Faktiske observationer & Godkendt
	\\[-1ex] \midrule
	&  \textit{Usability} &  &  & & \\ \midrule
	& Brugeren skal kunne starte en default-måling maksimalt 20 sekunder efter opstart af program & Start programmet, hvorefter der vha. stopur måles opstartstiden & At programmet er startet op indenfor 20 sekunder & & \\ \midrule
	& Brugeren skal have mulighed for at ændre tidsintervallet før målingerne foretages & Start programmet og ændrer indstillingerne i toolbar & At der er mulighed for at ændre indstillinger & & \\ \midrule
	& Login-vinduet skal indholde en login-knap til at logge på og få vist EKG-vinduet & login-knappen er synlig i GUI, og ved tryk på knappen vises EKG-vinduet & At EKG-vinduet vises & & \\ \midrule 
	& EKG-vinduet skal indeholde en "start-knap" til at igangsætte målingerne & Startknappen er synlig i GUI, og ved tryk på knappen igangsættes måling & At målingen igangsættes & & \\ \midrule
	& EKG-vinduet skal indeholde en "stop-knap" til at afslutte målingerne & Stopknappen er synlig i GUI, og ved tryk på knappen afsluttes måling & At målingen afsluttes  & & \\ \midrule
	& EKG-vinduet skal indeholde en "gem-knap" til at gemme målingerne & Gem-knappen er synlig i GUI, og ved tryk på knappen gemmes måling i database & Messageboks vises på skærmen med teksten "Måling er gemt" og kan findes i databasen & & \\ \midrule
	& EKG-vinduet skal indeholde en "log ud"-kanp til at logge ud & "log ud"-knappen er synlig i GUI, og ved tryk på knap lukkes EKG-vinduet og login-vinduet vises & Login-vinduet vises & & \\ \midrule
	& Målingen stopper automatisk efter det valgte tidsinterval & Der vælges et tidsinterval. Måling startes & Målingen stopper efter det valgte tidsinterval & & \\ \midrule
	& \textit{Reliability} & & & & \\ \midrule
	& Systemet skal have en effektiv MTBF (Mean Time Between Failure) på 20 minutter og MTTR (Mean Time To Restore) på 1 minut & Køre programmet i 20 minutter. Genstart derefter programmet, hvor der tages tid med et stopur & Programmet har kørt i 20 minutter og genstartes indenfor 1 minut  & & \\ \midrule
	& \textit{Performance} & & & & \\ \midrule
	& Der skal vises en EKG-graf i interfacet, hvor spænding vises op ad y-aksen (-1V til 1V) og tiden på x-aksen & Gennemfør en måling & At spændingen for EKG-signalet er op ad y-aksen, samt tiden hen ad x-aksen & & \\ \midrule
	& Det skal være muligt at kunne scrolle igennem målingerne hen ad x-aksen & Der gennemføres en måling hvorefter der scrolles hen ad x-aksen & At der ved scrolling kan ses forskellige dele af EKG-signalet hen ad x-aksen & & \\ \midrule
	& Der skal kunne tages et sample over et brugerbestemt interval, hvor frekvensen er tilpasses målingerne, således at grafen er analyserbar & Gennemfør en måling, hvor et bestemt interval hen ad x-aksen er valgt & At det valgte interval synliggøres & & \\ \midrule
	& \textit{Supportability} & & & & \\ \midrule
	& Softwaren er opbygget af trelagsmodellen & Kig i koden efter data-lag, logik-lag og GUI-lag & At koden indeholder et data-lag, et logik-lag og et GUI-lag & & \\ \bottomrule
\caption{Accepttest af Ikke-funktionelle krav}
\end{longtabu}

