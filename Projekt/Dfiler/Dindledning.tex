\chapter{Indledning}
Denne dokumentation udgør baggrunden for projektrapporten omkring EKG-måling af atrieflimren. I dokumentationen beskrives først og fremmest hovedscenariet, gennem udarbejdelse af kravspecifikation. \\
Kravspecifikationen består en beskrivelse af projektets funktionelle krav. Beskrivelse af aktører, samt hvordan de intergerer. Derudover også beskrivelse af de relevante use-cases samt systemets undtagelser. Desuden er systemets ikke-funktionelle krav også beskrevet, dette gennem systembeskrivelsesmetoden "(F)URPS+".\\ \\
Ydermere formidles en detaljeret beskrivelse af det komplette systems design, bestående af hardware og software arkitektur. Hardwaren arkitekturen beskriver den grundlæggende opstilling af systemet samt tilhørende grænseflader. Softwaren er beskrevet via en gennemgang af hovedforløbet illusterret ved skitser af GUI'en samt diverse relevante systembeskrivelsesmodeller. \\ \\
Afslutningsvis foretages en fyldestgørende accepttest, som primært har til formål at teste de opstillede funktionelle- og ikke-funktionelle krav. Accepttesten beskrives efterfølgende kronologisk fra use case 1 til ikke-funktionelle krav, for at i sidste ende at kunne dokumentere EKG-systemets funktionalitet.

\textbf{Ansvarsområde} \\
\textbf{Initialer: } \\
Albert Jakob Fredshavn - AJF \\
Cecilie Ammitzbøll Aarøe - CAA \\
Lise Skytte Brodersen - LSB \\
Martin Banasik - MBA \\
Malene Cecilie Mikkelsen - MCM \\
Mads Fryland Jørgensen - MFJ \\
Mohamed Hussain Mohamed - MHM \\
Sara-Sofie Staub Kirkeby - SSK \\

\begin{longtabu} to \linewidth{@{}  l X[j]@{}}
    Afsnit &    Ansvarlig\\[-1ex]
    \midrule
    Indledning & AJF og MFJ \\
    Kravspecifikation & LSB, AJF, CAA og MFJ\\
    Hardware arkitektur & LSB, SSK og MBA\\
    Software arkitektur & Alle\\
    Software implementering & SSK MHM\\
    Accepttest & LSB, AJF, CAA og MFJ\\
    Fejlrapport & MFJ\\
    
    
    

\end{longtabu}