\chapter{Indledning}
Denne dokumentation udgør baggrunden for projekt rapporten omkring EKG-måling af atrieflimren. I dokumentationen beskrives først og fremmest hovedscenariet, gennem udarbejdelse af kravspecifikation. \\
Kravspecifikationen består en beskrivelse af projektets funktionelle krav. Beskrivelse af aktører, samt hvordan de intergerer, derudover også beskrivelse af de relevante use-cases, samt systemets undtagelser. Desuden er systemets ikke-funktionelle krav også beskrevet, dette gennem systembeskrivelsesmetoden "(F)URPS+".\\ \\
Ydermere formidles en detaljeret beskrivelse af det komplette systems design, bestående af hardware og software arkitektur. Hardwaren arkitekturen beskriver den grundlæggende opstilling af systemet, samt tilhørende grænseflader. Softwaren er beskrevet via en gennemgang af hovedforløbet illusterret ved skitser af GUI'en, samt diverse relevante systembeskrivelsesmodeller. \\ \\
Afslutningsvis foretages en fyldestgørende accepttest, som primært af har til formål at teste de opstillede use-cases, samt systemets ikke-funktionelle krav. Accepttesten beskrives efterfølgende kronologisk fra use case 1 til ikke-funktionelle krav, for at i sidste ende at kunne dokumentere EKG-systemets funktionalitet.