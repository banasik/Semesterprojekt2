\chapter{Mødereferat}

\section{Dato: 11-02-2015}
\hrule

\textbf{Fremmødte:} Mads, Martin, Cecilie, Sara, Lise og Mohammed

\textbf{Fraværende:} Malene og Albert

\textbf{Dagens dagsorden:}
\begin{itemize}
	\item Hvordan kommer vi i gang? 
	\item Tidsplan - deadlines
	\item Spørgsmål
\end{itemize}

\textbf{Referat:}
\begin{itemize}
	\item Vi skal lave KS og AT for for-projektet
	\item Use Cases - funktionelle og ikke-funktionelle krav
	\item Der er nogle krav til, hvor vi skal måde i for-projektet (se diasshow fra intro-timen)
	\item Lav det på en person
\end{itemize} 

Sundhed: Vi skal måske skrive om, hvordan et rask hjerte fungere også gå over i, hvad fejlen er i hjertet til vores valgte sygdom indtræffer.  

Vi skal aflevere design og rapport hver for sig. 

LaTex: synes Lars vi skal tage til! 

Design har sin version historik - rapporten har - KS og AT har sin egen.  

Ugentlig møde : Onsdage kl. 11:30 - ellers send mail eller kig ind.  


\section{Dato: 13-02-2015}
\hrule

\textbf{Fremmødte:} Sara, Malene, Mads, Lise og Mohammed 

\textbf{Fraværende:} Albert, Cecilie og Martin

\textbf{Dagens dagsorden:}
\begin{itemize}
	\item Tidsplan
\end{itemize}

\textbf{Referat:}\\ 
Vi lavet tidsplan for hele projektet. Vi har aftalt videre omkring mødedisciplin. Mads og Lise har siddet efter selve mødet og leget med LaTex - vi fik lavet godkendelsesformularen i programmet! 

\textbf{Dagsorden for næste møde med vejleder:}
\begin{itemize}

	\item Skal samarbejdsaftalen med i rapporten som bilag? - svar bilag
	\item Gennemgang af tidsplan samt godkendelse 
	\item Hvad består design i? 
	\item Dokumentation, hvad er det? Og hvordan skal det indgår i selve rapporten? 
	\item Kig på LaTex dokument - kan vi bruge det som udgangspunkt?
	\item 6/3 reviewed eller aflevere til review
	\item Kig på use case
 
\end{itemize}

\section{Dato: 27-02-2015}
\hrule

\textbf{Fremmødte:} Martin, Mads, Albert, Sara, Mohammed, Malene og Lise 

\textbf{Fraværende:} Cecilie

\textbf{Dagens dagsorden:}
\begin{itemize}
	\item Lars skal kigge kravspec igennem
	\item Spørgsmål til ikke-funktionelle krav
\end{itemize}

\textbf{Referat:}

Kravspec er fin som udgangspunkt - skal skal slettes først punkt i Use Case 1 og i accepttesten. 

Spørgsmål til Reliability (ligning) - Lars viste ikke, hvad der var realistisk, så Albert ringede til sin mor (sygeplejerske) hun sagde : den kan holde i 5 år og 2 dages reparation.

Sara kan svare på de sidste punkter i performence.    

\section{Dato: 09-03-2015}
\hrule

\textbf{Fremmødte:} Sara, Malene, Martin, Mohamed, Albert, Mads og Lise 

\textbf{Fraværende:} Cecilie

\textbf{Dagens dagsorden:}
\begin{itemize}
	\item Review af gruppe 2's KS og AT
\end{itemize}

\textbf{Referat:}

Vi har gennemgået gruppe 2's KS og AT, hvor vi har skrevet kommentar, som vi fremlægger for dem på onsdag til mødet med Lars. 
Vi er også alle blevet introduceret til Github (dog mangler Cecilie). 
Vi kigger på sygdommene derhjemme fra og bestemmer, hvilken vi vil arbejde med næste gang! 

\section{Dato: 18-03-2015}
\hrule

\textbf{Fremmødte:} Martin, Albert, Muhammoud, Mads, Lise, Sara

\textbf{Fraværende:} Malene

\textbf{Dagens dagsorden:}

\begin{itemize}
	\item Hvad skal der ske i dag og resten af ugen? 
	\item Tidsplans rettelser. 
	\item Valg af sygdom
\end{itemize}

\textbf{Referat:}
Vi har rykket lidt i vores tidsplan, da vi har brugt lidt for længe på programmeringen til forprojektet. Vi har aftalt at programmeringsgruppen skal igang med diagrammerne til det rigtige projekt i næste uge, da det ellers bliver for meget forskudt. 

Vi har valgt at vores sygdom, som vi vil bruge i projektet, skal være artrieflimmer. 


\section{Dato: 20-03-2015}
\hrule

\textbf{Fremmødte:} Albert, Mohamed, Martin, Malene og Mads 

\textbf{Fraværende:} Lise, Cecilie og Sara

\textbf{Dagens dagsorden:}
\begin{itemize}
	\item Svar på spørgsmål omkring overgang fra forprojekt til hovedprojekt
\end{itemize}

\textbf{Referat: }\\ 
Design-mappe. Skal system test og acceptest ligge derinde?
De kan godt ligge der inde. \\
Skal vi teste ’Could’? 
MoSCoW kan bruges når man har mange use cases. Og kan bruges hvis der er vigtige dele der skal deles op i MoSCoW.
\\
Når man skriver det op i MosCow, burde der være en use case forbundet. 
\\
Skal vi sorterer nogle af dem fra (touch-skærm f.eks.)?
Den bør ikke være med i MoSCoW.
\\
Det der er ikke er funktionelle krav, skal ’bindes op’ på en use-case.
\\
Hvis DET er tidsmessigt af sammenhæng er det én usecase.
Hvis det ikke er tidsmessigt af sammenhæng er det to eller flere use cases.
\\
Vores EKG-system:
Aktør kontekst, kan vi lave en pil fra analog op til physionet?
Vi kan lave en ekstern aktør, som er physio-net.
\\
Hovedforløb i use-casen, hvordan skal det være i forhold til opkobling til patient/physionet
Som en forudsetning. 
Vi har et analog-signal, som skal opkobles til et EKG-signal.
\\
Aktør kontekst:
Vi skal have bruger i stedet for sundhedsprof.

\section{Dato: 17-04-2015}
\hrule

\textbf{Fremmødte:} Albert, Martin, Malene, Sara, Mohamed og Mads

\textbf{Fraværende:} Lise og Cecilie

\textbf{Dagens dagsorden:}
\begin{itemize}
	\item Respons på afsnit om baggrund
	\item respons på begyndende “sysML og UML”
	\item Generelt hjælp til programmering
	\item Hvad skal vi arbejde videre med i rapport?
\end{itemize}

\textbf{Referat:}Vi aftalte, at vi sender diagrammer til Lars og får derefter respons over mail, og hvis relevant, uddybet på næste møde. \\
	Respons på baggrunds afsnittet - trelagsmodel er ikke nødvendig at beskrive, kommer med i design gennem beskrivelse af vores system. \\
	Vi skal beskrive systemet med ord, gerne billede. Ikke om koden, men bare om hvordan det fungerer.\\
	Næste punkt er design \\
	For meget på SD - ikke analog og DAQ,  → blackbox\\
	Om EKG signalet, skal gemmes som rådata → gemme rådata, kan konverteres over i en liste + generelt vejledning til programmering \\
	Møde næste uge er rykket til onsdag kl 12 på 400 gangen - dagsorden sendes tirsdag, efter 	møde i hultimen

\section{Dato: 21-04-2015}
\hrule

\textbf{Fremmødte:} Martin, Cecilie, Mohamed, Mads, Albert og Lise 

\textbf{Fraværende:} Malene og Sara

\textbf{Dagens dagsorden:}
\begin{itemize}
	\item Lav dagsorden til møde med Lars onsdag 
\end{itemize}

\textbf{Referat:}
\\
Vi har fået lavet en dagsorden og sendt til Lars. 
Vi har aftalt at onsdag fra klokken 11-12 vil Martin lave et lille oplæg omkring programmet. Og klokken 12 har vi møde med Lars. 
Torsdag møder vi klokken 8 og arbejder med programmering og design.


\section{Dato: 22-04-2015}
\hrule

\textbf{Fremmødte:} Martin, Sara, Malene, Mads, Albert og Lise 

\textbf{Fraværende:} Mohamed og Cecilie

\textbf{Dagens dagsorden:}
\begin{itemize}
	\item Kommentar fra Lars
\end{itemize}

\textbf{Referat:}
\\
\\
\textbf{Kommentar til baggrund:} Det er en god ide, at referere til figurerne i teksten - ellers giver det ikke mening med billederne. 
\\
\\
\textbf{Kommentar til diagrammer:} Indsæt et billede af opstillingen af HW. 
Domæne er over alle UC's, mens klassediagram og sekvensdiagram skal laves over hver UC. 
Over algoritmen kan vi lave en aktivitetsdiagram. 
Trelagsmodel-diagrammet hedder en pakkediagram. Kan også laves om til UML-klassediagram - HUSK pile. 

\section{Dato: 29-04-2015}
\hrule


\textbf{Fremmødte:} Mads, Albert, Mohamed, Martin, Sara, Malene og Lise

\textbf{Fraværende:} Cecilie

\textbf{Dagens dagsorden:}
\begin{itemize}
	\item Feedback fra Lars på diagrammer samt systemarkitektur
\end{itemize}

\textbf{Referat:}

BDD er irrelevant, billedet erstatter
\\ 
Spg: skal knapper også ind I UML - ikke nødvendigt at have \\
	Skal skrives i design overvejelser
\\

Spg: hvor vælger man liste under EKG (i visio)?\\
	De skal skrives ind i design overvejelser
\\
Note prog.: I skal selv lave typen (af listen el. whatever)
\\
Spg: Blackbox, hvordan håndterer man den?\\
	Der skal være et interface, properties, som kan sættes på
\\
Spg: ”Using national instruments”, skal vi have den med?\\
	Den kode fra Jesper, skal betraktes som 3. party software.
	Man skal beskrive det interface, som Blackbox/Jespers Kode/Hvad den nu hedder\\
	Det er kun set udefra som er i den\\
	Henvis kun til Jespers kode.
\\
Spg: Vil du gerne have at hver lag beskrives hver for sig? Eller skal de ind under designovervejelser?\\
Man kan lave et pakke-diagram, men når vi har lavet de andre diagrammer er det ikke nødvendigt.\\
De forskellige lag skal beskrives i designovervejelsen til UML diagrammet
\\
Rapporten: Meget fakta og præsenteret som gruppens arbejde, ikke individuelle kapitler.\\
	Derfor: skal omformuleres (det er en god øvelser, det skal gøres upersonligt)
	Niggas in trainin’
\\
Spg: Sekvensdiagram 1
	Metodekald, besked tilbage.	\\
Ikke metodenavn (navn med parentes bag på) på ”Bekræft login”\\
Spørg Kim\\
Ellers ser Sekvens diagrammet fint ud
\\

Sekvensdiagram 2:\\
	Reelt skal der stå: Objektet
\\
Sekvensdiagrammer generelt\\
	Hvis den sender besked op til at åbne, sender den besked til GUI\\
	Det hele skal gå igennem controller-klassen (som ikke er i logik-laget)
\\

Vores controller skal have ”åbner” metoden.
Der hvor pilene går hen, skal metoden åbnes i\\
Ses evt. Kim’s Metodekalds-beskrivelse
\\

De skærmbilleder der er vist er skitser.


\section{Dato: 12-05-2015}
\hrule


\textbf{Fremmødte:} Malene, Sara, Mads, Albert, Mouhammed og Lise 

\textbf{Fraværende:} Martin

\textbf{Dagens dagsorden:}
\begin{itemize}
	\item Hans tanker omkring databse og disse problemer
	\item Hans kommentarer på database
	\item Spørgsmål til dokumentationsrapport
\end{itemize}

\textbf{Referat:}

Der er fikset et mindre problem med Databasen. 

Der er blevet rettet nogen fejl i designdelen. I UML-diagrammet skal de metoder som bliver brugt, skrives i UML'en. Der skal laves et seperat afsnit hvor analysen bliver gennemgået i detaljer. Der skal skrives noget om implementering i dokumentationsrapporten. Her skal analysen og testprogrammet ligge. 

Der behøves ikke lave dokumentation om offentlig database. 

Der køres acceptest onsdag klokken 14:30. 

\section{Dato: 20-05-2015}
\hrule

\textbf{Fremmødte: } Malene, Sara, Lise, Cecilie og Mads

\textit{Dagens dagsorden:}
\begin{itemize}
	\item Acceptest med vejleder
\end{itemize}
Accepttesten er gennemgået og godkendt af Lars.
Fejl skal skrives i en fejlrapport under bilag.

 


