\chapter{Mødereferat}

\textbf{Dato:} 29-04-2015

\textbf{Fremmødte:} Mads, Albert, Mohamed, Martin, Sara, Malene og Lise

\textbf{Fraværende:} Cecilie

\textbf{Dagens dagsorden:}
\begin{itemize}
	\item Feedback fra Lars på diagrammer samt systemarkitektur
\end{itemize}

\textbf{Referat:}

BDD er irrelevant, billedet erstatter
\\ \\
Spg: skal knapper også ind I UML - ikke nødvendigt at have \\
	Skal skrives i design overvejelser
\\
\\

Spg: hvor vælger man liste under EKG (i visio)?\\
	De skal skrives ind i design overvejelser
\\
\\

Note prog.: I skal selv lave typen (af listen el. whatever)
\\
\\

Spg: Blackbox, hvordan håndterer man den?\\
	Der skal være et interface, properties, som kan sættes på
\\
\\
	
Spg: ”Using national instruments”, skal vi have den med?\\
	Den kode fra Jesper, skal betraktes som 3. party software.
	Man skal beskrive det interface, som Blackbox/Jespers Kode/Hvad den nu hedder\\
	Det er kun set udefra som er i den\\
	Henvis kun til Jespers kode.
\\
\\

Spg: Vil du gerne have at hver lag beskrives hver for sig? Eller skal de ind under designovervejelser?\\
Man kan lave et pakke-diagram, men når vi har lavet de andre diagrammer er det ikke nødvendigt.\\
De forskellige lag skal beskrives i designovervejelsen til UML diagrammet
\\
\\

Rapporten: Meget fakta og præsenteret som gruppens arbejde, ikke individuelle kapitler.\\
	Derfor: skal omformuleres (det er en god øvelser, det skal gøres upersonligt)
	Niggas in trainin’
\\
\\
Spg: Sekvensdiagram 1
	Metodekald, besked tilbage.	\\
Ikke metodenavn (navn med parentes bag på) på ”Bekræft login”\\
Spørg Kim\\
Ellers ser Sekvens diagrammet fint ud
\\
\\

Sekvensdiagram 2:\\
	Reelt skal der stå: Objektet
\\
\\
Sekvensdiagrammer generelt\\
	Hvis den sender besked op til at åbne, sender den besked til GUI\\
	Det hele skal gå igennem controller-klassen (som ikke er i logik-laget)
\\
\\

Vores controller skal have ”åbner” metoden.
Der hvor pilene går hen, skal metoden åbnes i\\
Ses evt. Kim’s Metodekalds-beskrivelse
\\
\\

De skærmbilleder der er vidst er skitser.
