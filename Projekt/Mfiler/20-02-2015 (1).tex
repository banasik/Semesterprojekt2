\chapter{Mødereferat}

\textbf{Dato:} 20-02-2015

\textbf{Fremmødte:} Lise, Malene, Mads, Albert, Sara og Mohammed. 

\textbf{Fraværende:} Martin

\textbf{Dagens dagsorden:} 
\begin{itemize}
	\item Gennemgang af spørgsmål.
	\item Gennemgang af tidsplan. 
\end{itemize}

\textbf{Referat:}
I dag har alle, undtagen Martin, underskrevet samarbejdssamtalen. Vi har delt op i hold, og begyndt at arbejde på forholdsvist programmeringen, og kravsspec/Accepttest. 
\\Samarbejdssamtalen skal være en del af bilag – ikke rapporten. 
\\Der skal afsættes tid til at lave review af den anden gruppe. 
\\God tidsplan. 
\\I design delen skal der være domænemodel, systemarkitektur, diagrammer – alt det fis. 
\\Deserialisering – data bliver pakket ind objekt. Vi gemmer data’erne og viser den. Svarer til data lag. 
\\Start datalag med – vent til efter programmerings teori. Eventuel en masse selvstudie. 
\\Man behøver ikke at få 100 procent styr på sourcecoden. Brug blackbox conceptet. 
\\Den anden del af gruppen er gået i gang med at lave aktør kontekst diagrammer. Den får en thumbs up. 
\\Til forprojektet er der kun lavet en usecase. Evaluer EKG resultat med HVR. Se Samuels slides. 
\\Undtagelser; Skal kun være hvis der sker fejl. Hvis forbindelsen forsvinder – hvordan reagerer kurven? Det er en fin undtagelse. Endnu et thumbs up. 
\\Gratis program – StarUML. Alternativ til Visio. 
\\Forskel på logbog og referat; Hvis vi sidder og arbejder, så er det logbog. Hvis det er et officielt møde, så skal der skrives logbog.
\\Møde igen fredag klokken 12. 


