%%%% Dokumentklassen %%%%

\documentclass[a4paper,11pt,fleqn,dvipsnames,twoside,openright]{memoir} 	% Openright åbner kapitler på højresider (openany begge)


%%%% PACKAGES %%%%

%% Oversættelse og tegnsætning %%
\usepackage[utf8]{inputenc}					% Input-indkodning af tegnsæt (UTF8)
\usepackage[danish]{babel}					% Dokumentets sprog
\usepackage[T1]{fontenc}					    % Output-indkodning af tegnsæt (T1)
\usepackage{ragged2e,anyfontsize}			% Justering af elementer
\usepackage{fixltx2e}						% Retter forskellige fejl i LaTeX-kernen
\usepackage{lastpage} 						% antal sider i dukumentet
																			
%% Figurer og tabeller (floats) %%
\usepackage{graphicx} 						% Håndtering af eksterne billeder (JPG, PNG, EPS, PDF)
\usepackage{multicol}         	            	% Muliggør output i spalter
\usepackage{rotating}						% Rotation af tekst med \begin{sideways}...\end{sideways}
\usepackage{xcolor}							% Definer farver med \definecolor. Se mere: http://en.wikibooks.org/wiki/LaTeX/Colors
\usepackage{flafter}						% Sørger for at floats ikke optræder i teksten før deres reference
\let\newfloat\relax 						% Justering mellem float-pakken og memoir
\usepackage{float}							% Muliggør eksakt placering af floats, f.eks. \begin{figure}[H]

%% Matematik mm. %%
\usepackage{amsmath,amssymb,stmaryrd} 		% Avancerede matematik-udvidelser
\usepackage{mathtools}						% Andre matematik- og tegnudvidelser
\usepackage{textcomp}                 		% Symbol-udvidelser (fx promille-tegn med \textperthousand)
\usepackage{rsphrase}						% Kemi-pakke til RS-saetninger, fx \rsphrase{R1}
\usepackage[version=3]{mhchem} 				% Kemi-pakke til flot og let notation af formler, f.eks. \ce{Fe2O3}
\usepackage{siunitx}						% Flot og konsistent præsentation af tal og enheder med \si{enhed} og \SI{tal}{enhed}
\sisetup{output-decimal-marker = {,}}		% Opsætning af \SI (DE for komma som decimalseparator) 

%% Referencer og kilder %%
\usepackage[danish]{varioref}				% Muliggør bl.a. krydshenvisninger med sidetal (\vref)
\usepackage{natbib}							% Udvidelse med naturvidenskabelige citationsmodeller
\usepackage{xr}							    % Referencer til eksternt dokument med \externaldocument{<NAVN>}

%% Misc. %%
\usepackage{listings}						% Placer kildekode i dokumentet med \begin{lstlisting}...\end{lstlisting}
\usepackage{lipsum}							% Dummy text \lipsum[..]
\usepackage[shortlabels]{enumitem}			% Muliggør enkelt konfiguration af lister
\usepackage{pdfpages}						% Gør det muligt at inkludere pdf-dokumenter med kommandoen \includepdf[pages={x-y}]{fil.pdf}	
\pdfoptionpdfminorversion=6					% Muliggør inkludering af pdf-dokumenter, af version 1.6 og højere
\pretolerance=2500 							% Justering af afstand mellem ord (højt tal, mindre orddeling og mere luft mellem ord)	


%%%% CUSTOM SETTINGS %%%%

%% Marginer %%
\setlrmarginsandblock{3.5cm}{2.5cm}{*}		% \setlrmarginsandblock{Indbinding}{Kant}{Ratio}
\setulmarginsandblock{2.5cm}{3.0cm}{*}		% \setulmarginsandblock{Top}{Bund}{Ratio}
\checkandfixthelayout 						% Oversætter værdier til brug for andre pakker

%% Afsnitsformatering %%
\setlength{\parindent}{0mm}           		% Størrelse af indryk
\setlength{\parskip}{3mm}          			% Afstand mellem afsnit ved brug af double Enter
\linespread{1,1}							% Linjeafstand

%% Indholdsfortegnelse %%
\setsecnumdepth{subsection}		 			% Dybden af nummererede overskrifter (part/chapter/section/subsection)
\maxsecnumdepth{subsection}					% Dokumentklassens grænse for nummereringsdybde
\settocdepth{subsection} 					% Dybden af indholdsfortegnelsen
		
%% Opsætning af listings %%
\definecolor{commentGreen}{RGB}{34,139,24}
\definecolor{stringPurple}{RGB}{208,76,239}

\lstset{language=Matlab,					    % Sprog
	basicstyle=\ttfamily\scriptsize,		    % Opsætning af teksten
	keywords={for,if,while,else,elseif,		% Nøgleord at fremhæve
			  end,break,return,case,
			  switch,function},
	keywordstyle=\color{blue},				% Opsætning af nøgleord
	commentstyle=\color{commentGreen},		% Opsætning af kommentarer
	stringstyle=\color{stringPurple},		% Opsætning af strenge
	showstringspaces=false,					% Mellemrum i strenge enten vist eller blanke
	numbers=left, numberstyle=\tiny,		    % Linjenumre
	extendedchars=true, 					    % Tillader specielle karakterer
	columns=flexible,						% Kolonnejustering
	breaklines, breakatwhitespace=true,		% Bryd lange linjer
}

%% Navngivning %%
\addto\captionsdanish{
	\renewcommand\appendixname{Appendiks}
	\renewcommand\contentsname{Indholdsfortegnelse}	
	\renewcommand\appendixpagename{Appendiks}
	\renewcommand\appendixtocname{Appendiks}
	\renewcommand\cftchaptername{\chaptername~}		% Skriver "Kapitel" foran kapitlerne i indholdsfortegnelsen
	\renewcommand\cftappendixname{\appendixname~}	% Skriver "Appendiks" foran appendiks i indholdsfortegnelsen
}

%% Kapiteludssende %%
\definecolor{numbercolor}{gray}{0.7}		            % Definerer en farve til brug til kapiteludseende
\newif\ifchapternonum

\makechapterstyle{jenor}{					        % Definerer kapiteludseende frem til ...
  \renewcommand\beforechapskip{0pt}
  \renewcommand\printchaptername{}
  \renewcommand\printchapternum{}
  \renewcommand\printchapternonum{\chapternonumtrue}
  \renewcommand\chaptitlefont{\fontfamily{pbk}\fontseries{db}\fontshape{n}\fontsize{25}{35}\selectfont\raggedleft}
  \renewcommand\chapnumfont{\fontfamily{pbk}\fontseries{m}\fontshape{n}\fontsize{1in}{0in}\selectfont\color{numbercolor}}
  \renewcommand\printchaptertitle[1]{%
    \noindent
    \ifchapternonum
    \begin{tabularx}{\textwidth}{X}
    {\let\\\newline\chaptitlefont ##1\par} 
    \end{tabularx}
    \par\vskip-2.5mm\hrule
    \else
    \begin{tabularx}{\textwidth}{Xl}
    {\parbox[b]{\linewidth}{\chaptitlefont ##1}} & \raisebox{-15pt}{\chapnumfont \thechapter}
    \end{tabularx}
    \par\vskip2mm\hrule
    \fi
  }
}											        % ... her

\chapterstyle{jenor}						        % Valg af kapiteludseende - Google 'memoir chapter styles' for alternativer

%% Sidehoved %%

\makepagestyle{AAU}							        % Definerer sidehoved og sidefod udseende frem til ...
\makepsmarks{AAU}{%
	\createmark{chapter}{left}{shownumber}{}{. \ }
	\createmark{section}{right}{shownumber}{}{. \ }
	\createplainmark{toc}{both}{\contentsname}
	\createplainmark{lof}{both}{\listfigurename}
	\createplainmark{lot}{both}{\listtablename}
	\createplainmark{bib}{both}{\bibname}
	\createplainmark{index}{both}{\indexname}
	\createplainmark{glossary}{both}{\glossaryname}
}
\nouppercaseheads									% Ingen Caps ønskes

\makeevenhead{AAU}{\small ST2PRJ2 Gruppe 1}{}{\leftmark}	% Definerer lige siders sidehoved (\makeevenhead{Navn}{Venstre}{Center}{Hoejre})
\makeoddhead{AAU}{\rightmark}{}{\small ASE}		            % Definerer ulige siders sidehoved (\makeoddhead{Navn}{Venstre}{Center}{Højre})
\makeevenfoot{AAU}{\small \thepage}{}{}						% Definerer lige siders sidefod (\makeevenfoot{Navn}{Venstre}{Center}{Højre})
\makeoddfoot{AAU}{}{}{\small \thepage}						% Definerer ulige siders sidefod (\makeoddfoot{Navn}{Venstre}{Center}{Højre})

\copypagestyle{AAUchap}{AAU}							% Sidehoved for kapitelsider defineres som standardsider, men med blank sidehoved
\makeoddhead{AAUchap}{}{}{}
\makeevenhead{AAUchap}{}{}{}
\makeheadrule{AAUchap}{\textwidth}{0pt}
\aliaspagestyle{chapter}{AAUchap}					% Den ny style vælges til at gælde for chapters
													% ... her
															
\pagestyle{AAU}										% Valg af sidehoved og sidefod


%%%% CUSTOM COMMANDS %%%%

%% Billede hack %%
\newcommand{\figur}[4]{
		\begin{figure}[H] \centering
			\includegraphics[width=#1\textwidth]{billeder/#2}
			\caption{#3}\label{#4}
		\end{figure} 
}

%% Specielle tegn %%
\newcommand{\decC}{^{\circ}\text{C}}
\newcommand{\dec}{^{\circ}}
\newcommand{\m}{\cdot}


%%%% ORDDELING %%%%

\hyphenation{}


%%%% Tilføjelser af min preample %%%%

% Booktabs:
% The booktabs package is needed for better looking tables. 
\usepackage{booktabs}

% Caption:
% For better looking captions. See caption documentation on how to change the format of the captions.
\usepackage[hang, font={small, it}]{caption}

% Hyperref:
% This package makes all references within your document clickable. By default, these references will become boxed and colored. This is turned back to normal with the \hypersetup command below.
\usepackage{hyperref}
	\hypersetup{colorlinks=false,pdfborder=0 0 0}

% Cleveref:
% This package automatically detects the type of reference (equation, table, etc.) when the \cref{} command is used. It then adds a word in front of the reference, i.e. Fig. in front of a reference to a figure. With the \crefname{}{}{} command, these words may be changed.
\usepackage{cleveref}
	\crefname{equation}{formel}{formler}
	\crefname{figure}{figur}{figurer}	
	\crefname{table}{tabel}{tabeller}

% Mine tilføjelser:
\usepackage{units}                        %% Bruges til at gøre fx 1/2 samlet med: \nicefrac{1}{2}.
\usepackage{tabu, longtable}              %% Bruges til tabeller.
\setlength{\tabulinesep}{1.5ex}           %% Definerer linjeafstand i tabeller.
\usepackage{enumerate}                    %% Bruges til lister.
\usepackage{tabto}                        %% Giver mulighed for TAB med fx \tabto{3em}.
\usepackage[hyphenbreaks]{breakurl}       %% Bruges til websiders url'er.
\renewcommand{\UrlFont}{                  %% Definerer url-font.
\small\ttfamily}                          %
\bibliographystyle{plain}                 %% Definere bibliografien. Ses til sidst i dokumentet i kapitlet Litteratur.