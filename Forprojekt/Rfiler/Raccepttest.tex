\chapter{Accepttest}

\section{Accepttest af Use Cases}


%Use case 1 acceptest
\subsection{Use Case 1}



\textbf{Testopstilling for Use Case 1}
\begin{enumerate}
	\item DAQ’en har via USB-indgang forbindelse til computeren.
	\item DAQ’en er tilsluttet EKG-forstærkeren.
\end{enumerate}

\begin{longtabu} to \linewidth{@{} c X[j] X[j] X[j] l@{}}
    ~ &	Test &    Forventet resultat &		Faktiske observationer &    Godkendt\\[-1ex]
    \midrule
    ~ &\textit{Hovedsenarie} & ~ & ~ &
    \\ \midrule
    1. &Den sundhedsprofessionelle vælger tidsindstillinger &    Der er blevet valgt tidsindstillinger  &    ~ &		%{\Huge \checkmark}
    \\
    2. &Målingen startes ved at trykke på "Start" &    EKG-system indlæser data fra elektroderne  &     ~ &		%{\Huge \checkmark}
    \\
   	3. &EKG-dataerne illustreres på en graf  &    En analyserbar graf forekommer &    ~ &		%{\Huge \checkmark}
	\\ \midrule
	~ &\textit{Exentions} & ~ & ~ & 
	\\ \midrule	
    1a. &	Den sundhedsprofessionelle er tilfreds med default tidsindstillingerne
 &    Der blev ikke ændret i tidsindstillingerne  &   ~  &		%{\Huge \checkmark}
 \\ \bottomrule
 
\caption{Accepttest af Use Case 1.}\\
\label{AT_UC1}
\end{longtabu}

%Use Case 2 acceptest

\subsection{Use Case 2}

\textbf{Testopstilling for Use Case 2}
\begin{enumerate}
	\item DAQ’en har via USB-indgang forbindelse til computeren.
	\item DAQ’en er tilsluttet EKG-forstærkeren.
	\item Use case 1 er gennemført og graf forekommer på skærm
\end{enumerate}

\begin{longtabu} to \linewidth{@{} c X[j] X[j] X[j] l@{}}
    ~ &	Test &    Forventet resultat &		Faktiske observationer &    Godkendt\\[-1ex]
    \midrule
    ~ &\textit{Hovedsenarie} & ~ & ~ &
    \\ \midrule
    1. &Sundhedsprofessionelle observerer grafen &    Det er muligt at se forskelle i RR-intervallerne, samt takke og puls  &    ~ &		%{\Huge \checkmark}
    \\
    2. &Den sundhedsprofessionelle analyserer grafen ud fra 		  HRV	 &    HRV er synligt og grafen er fuldendt  &     ~ &		%{\Huge \checkmark}
	\\ \midrule
	~ &\textit{Exentions} & ~ & ~ & 
	\\ \midrule	
    2a. &	Det er ikke muligt at aflæse HRV på graf &    Der blev ikke evalueret på EKG  &   ~  &		%{\Huge \checkmark}
 \\ \bottomrule
 
\caption{Accepttest af Use Case 2.}\\
\label{AT_UC2}
\end{longtabu}


\subsection{Ikke-funktionelle krav}

\begin{longtabu} to \linewidth{@{} X[j] X[j] X[j] l@{}}
    Ikke-funktionelt krav &    Test/Handling & Forventet resultat &		Faktiske observationer\\[-1ex]
    \midrule
    \textit{Usability} & ~ & ~ & ~ 
    \\ \midrule
    	Den sundhedsprofessionelle skal kunne starte en default-måling maksimalt 20 sek. efter opstart af programmet &
        Trykker start hvorefter der vha. Stopur tages tid  &    
        At programmet er startet op inden 20 sekunder &	
        ~ 	%{\Huge \checkmark}
    \\
    	Den sundhedsprofessionelle skal have mulighed for at ændre tidsintervallet før målingerne foretages &
        Starter programmet og ændrer indstillinger i toolbar  &    
        At det er muligt at ændre indstillinger &	
        ~ 		%{\Huge \checkmark}
    \\
		Interfacet skal indeholde en "start-knap til at igangsætte målingerne &
		Startknappen kan ses i GUI, og ved klik startes måling &
		Der er en start knap, som starter måling når der trykkes &
		~ 		%{\Huge \checkmark}
	\\
		Interfacet skal indeholde en "stop-knap til at afslutte målingerne før den valgte tid
		Stopknappen kan ses i GUI, og ved klik stoppes igangværende måling &
		Der er en stopknap, som stopper igangværende måling når der trykkes &
		~ 		%{\Huge \checkmark}
	\\
		Programmet stopper automatisk efter det valgte tidsinterval &
		Der er valgt et tidsinterval - programmet kører tidsintervallet ud &
		Når tidsintervallet er gået stopper programmet &
		~ 	%{\Huge \checkmark}
		\\ \midrule	
    \textit{Reliability} & ~ & ~ & ~ 
    \\ \midrule
    	Systemet skal have en effektiv MTBF (Mean Time Between Failure) på 20 minutter og en MTTR (Mean Time To Restore) på 1 minut &
    	Programmet kører i 20 min - herefter genstartes, hvor der tages tid &
    	At availability er < 95,2 \%, som følge af availability formlen (reference?) &
    		%{\Huge \checkmark}
    \\ \midrule	
  	\textit{Performance} & ~ & ~ & ~
    \\ \midrule
    	Der skal vises en EKG-graf i interfacet, hvor spænding vises op af y-aksen (-1V til 1V) og tiden på x-aksen &
    	Der gennemføres en måling &
    	At spændingen for ekg-signalet er op ad y-aksen, samt tiden hen af x-aksen &
    		%{\Huge \checkmark}
    \\
    	Grafen skal have major gridlines hver 0,5 mV og minor gridlines hver 0,1 mV på y-aksen og major gridlines hver 200 ms. og minor gridlines hver 40 ms. på x-aksen &
    	Der gennemføres en måling &
    	Grafen har major (på 0,5 mV) og minor gridlines (på 0,1 mV) på y-aksen , samt major (på 200 ms) og minor (på 40 ms) gridlines på x-aksen &
    		%{\Huge \checkmark}
    \\
    	Grafen skal være scrollbar på x-aksen, så den sundhedsprofessionelle selv ved brug af musen kan vælge det udsnit af grafen der skal vises mere detaljeret &
    	Der gennemføres en måling, hvorefter bruger scroller hen ad x-aksen &
    	At der ved scrolling kan ses forskellige dele af EKG-signalet hen ad x-aksen &
    		%{\Huge \checkmark}
    \\
    	Skal tage en sample over et brugerbestemt interval, hvor frekvensen er tilpasset målingerne, således at grafen er analyserbar &
    	Der foretages en måling, hvor et bestemt interval hen ad x-aksen  vælges. &
    	At det valgte interval synliggøres
    		%{\Huge \checkmark}
 \\ \bottomrule
 
\caption{Accepttest af Ikke-funktionelle krav}
\label{AT_Ikke-funktionelle krav}
\end{longtabu}


