\chapter{Kravspecifikation}

\section{Indledning}
Kravspecifikationen vil beskrive, ud fra en række modeller, hvordan EKG-systemet fungerer. Helt generelt er EKG-måling en simpel metode, til at måle hjertets elektriske aktivitet via elektroder, som registrerer elektriske impulser, placeret på huden. Ud fra disse impulser dannes en graf, som benyttes til at analysere hjertets funktionalitet ud fra P-, Q-, R-, S- og T-takkerne, og dermed konkludere om den pågældende patient har et raskt eller sygt hjerte, samt hvilken sygdom der er tale om.

\section{Funktionelle krav}
De funktionelle krav vil nedenstående beskrives ud fra Aktør-kontekstdiagram, aktørbeskrivelse, Use Cases samt Use Case diagram. 

\subsection{Aktør-kontekstdiagram}

\begin{figure}[htb]
	\centering
	\includegraphics[width=1\textwidth]{Figurer/Snip20150313_4}
	\caption{Aktør-kontekstdiagram}
	\label{fig:aktoerbeskrivelse}
\end{figure}

Patient kobles op med elektroderne fra EKG-systemet (jf. use case 1). Patientens data bliver opsamlet af EKG-systemet, ud fra disse data danner EKG-systemet en graf. Grafen kan derefter valideres og analyseres af den sundhedsprofessionelle. 

\subsection{Aktørbeskrivelse}

\begin{table}[H]
\begin{tabularx}{\textwidth}{l l X}
     Aktørnavn  & Type      & Beskrivelse \\ \midrule
     Sundhedsprofessionelle   & Primær    & Det er den sundhedsprofessionelle, der ønsker at foretage EKG-målinger samt analysere på EKG-grafen.\\ 						  									  \addlinespace[2mm]
     Patienten & Sekundær  & Patienten udsender hjerteimpulser, som opfanges af elektroner, der er placeret på patientens krop.\\                                                                                                                                                                            
   
     \bottomrule                                                                                                                   
    \end{tabularx}
    \caption {Aktørbeskrivelse}
    \label{tab:aktoerbeskrivelse}
	
\end{table}

\subsection{Use Cases}

\begin{longtabu} to \linewidth{@{}l r X[j]@{}} %UC1%
    {\large \textbf{Use Case 1}} && \\
    \toprule
    Navn &&    Vis EKG\\
    Use case ID &&    1\\
    Samtidige forløb &&    1\\
    Primær aktør &&    Den sundhedsprofessionelle\\
    Sekundær aktør &&	Patienten\\
    Initialisere &&    Den sundhedsprofessionelle ønsker at få vist et EKG-signal over patienten\\
    Forudsætninger &&    EKG-elektroder er koblet rigtigt op på patienten ud fra afledning II :
     \begin{itemize}
     	\item Rød(+) på venstre ben
     	\item Sort(-) på højre arm
     	\item Grøn på venstre arm
     \end{itemize}
		Samt EKG-systemet er tændt og klar til måling\\
    Resultat &&    Den sundhedsprofessionelle kan ud fra EKG-dataerne se en graf                     \\ \midrule
    Hovedforløb &    1. &    Den sundhedsprofessionelle vælger indstillinger\newline
    						 [1.a \textit{Den sundhedsprofessionelle er tilfreds med default-indstillingerne}]\\[-1ex]	
                &    2. &    Målingen startes ved at trykke på "Start"\\[-1ex]
                &    3. &    EKG-dataerne illustreres på en graf\\ \midrule
                
    Undtagelser &    1a. &    Der blev ikke ændret i indstillingerne. Der fortsættes ved punkt 2 i hovedforløbet med default indstillingerne \\ \bottomrule
\caption{Fully dressed Use Case 1.}
\label{UC1}
\end{longtabu}

\begin{longtabu} to \linewidth{@{}l r X[j]@{}} %UC2%
    {\large \textbf{Use Case 2}} && \\
    \toprule
    Navn &&    Evaluer EKG i forhold til HRV\\
    Use case ID &&    2\\
    Samtidige forløb &&    1\\
    Primær aktør &&    Den sundhedsprofessionelle\\
    Initialisere &&    Use Case 1 er gennemført   \\
    Resultat &&   HRV kan ses ud fra grafen \\ \midrule
    Hovedforløb &    1. &    Den sundhedsprofessionelle måler længden på RR intervallerne\\[-1ex]	
                &    2. &    Den sundhedsprofessionelle analyserer målingerne\\[-1ex]
                &    3. &    HRV er identificeret\newline
                			 [3.a \textit{HRV er ikke identificerbart}]\\ \midrule
                
    Undtagelser &    3a. &    Det er ikke muligt at analysere HRV ud fra grafen. Use case 2 afsluttes og Use case 1 gentages med evt. nye tidsindstillinger \\ \bottomrule
\caption{Fully dressed Use Case 2.}
\label{UC2}
\end{longtabu}

\subsection{Use case-diagram}

\begin{figure}[htb]
	\centering
	\includegraphics[width=1\textwidth]{Figurer/Snip20150226_2}
	\caption{Use case-diagram}
	\label{fig:Use Cases}
\end{figure}

Den sekundære aktør, Patienten, er koblet op til elektroderne og EKG-systemet, som på kommando af den primære aktør, den sundhedsprofesionelle, danner en graf (Uc1: Vis EKG). Denne graf kan den sundhedsprofessionelle efterfølgende evaluere (Uc2: Evaluer EKG).

\section{Ikke-funktionelle krav}
De ikke-funktionelle krav er udarbejdet ved brug af (F)URPS+. De er alle prioriteret ved MoSCoW metoden - Must (skal være med), Should (bør være med, hvis muligt), Could (kunne have med, hvis det ikke influerer på andet), Won't/Would (ikke med nu, men med i fremtidige opdateringer). 

\subsection{(F)URPS+}
MoSCoW er angivet i parentes med hhv. M, S, C eller W.

\textbf{Usability}
\begin{itemize}
	\item (M) Den sundhedsprofessionelle skal kunne starte en default-måling maksimalt 20 sek. efter opstart af programmet
	\item (M) Den sundhedsprofessionelle skal have mulighed for at ændre tidsintervallet før målingerne foretages
	\item (M) Interfacet skal indeholde en "start"-knap til at igangsætte målingerne
	\item (M) Interfacet skal indeholde en "stop"-knap til at afslutte målingerne før den valgte tid
	\item (M) Programmet stopper automatisk efter det valgte tidsinterval
	\item (C) Programmet kan indeholde "pause/stop-knap"
	\item (S) Interfacet bør anvendes på en touch-skærm. Dette gør den nemmere at rengøre og simplere at anvende
	\item (S) Der bør kræves et login i form af patientens cpr-nummer inden opstart af programmet
\end{itemize}

\textbf{Reliability}
\begin{itemize}
	\item (S) Softwaren skal opdateres to gange årligt
	\item (M) Systemet skal have en effektiv MTBF (Mean Time Between Failure) på 20 minutter og en MTTR (Mean Time To Restore) på 1 minut.
	\item  
				\begin{align}
					Availability = \frac{MTBF}{MTBF+MTTR} = \frac{20}{20+1} = 0,952 = 95,2 \%
				\end{align}

\end{itemize}

\textbf{Performance}
\begin{itemize}
	\item (M) Der skal vises en EKG-graf i interfacet, hvor spænding vises op af y-aksen (-1V til 1V) og tiden på x-aksen
	\item (M) Grafen skal have major gridlines hver 0,5 mV og minor gridlines hver 0,1 mV på y-aksen og major gridlines hver 200 ms. og minor gridlines hver 40 ms. på x-aksen
	\item (M) Grafen skal være scrollbar på x-aksen, så den sundhedsprofessionelle selv ved brug af musen kan vælge det udsnit af grafen der skal vises mere detaljeret
	\item (S) Det er ønskeligt hvis en 1 mV signal-tak kan vises i starten af grafen som reference for det målte EKG-signal
	\item (M) Skal tage en sample over et brugerbestemt interval, hvor frekvensen  er tilpasset målingerne, således at grafen er analyserbar
\end{itemize}

\textbf{Supportability}
\begin{itemize}
	\item (M) Softwaren udarbejdes i Visual Studio
	\item (M) Softwaren er opbygget af trelagsmodellen
	\item (C) Systemet skal selv kunne søge efter opdateringer. Den skal selv kunne opdateres såfremt det ikke påvirker målingerne
\end{itemize}














